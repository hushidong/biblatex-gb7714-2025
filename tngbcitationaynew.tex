
% !Mode:: "TeX:UTF-8"
% 用于测试gb7714-2015ay样式,实现GB/T 7714-2015 标准说明中给出的著者年份制示例
\documentclass{article}
\usepackage{ctex}
\usepackage{xcolor}
\usepackage{toolbox}
\usepackage[colorlinks]{hyperref}
\usepackage{lipsum}
\usepackage[a4paper,top=3cm,bottom=2cm,left=2.54cm,right=2.54cm]{geometry}
\usepackage{xltxtra,mflogo,texnames}
\usepackage[backend=biber,style=gb7714-2025ay,gbnoauthor,gbpub=true,
sorting=none
%mergedate=none
%,sortlocale=zh__stroke,nosortothers=true,nohashothers=true,
]{biblatex}%sorting=nyt


\addbibresource{example2025.bib}

\setcounter{secnumdepth}{4}
\usepackage{titlesec}
\titleformat{\section}{\color{white}}{\thesection}{1em}{}
%\titlespacing{\section}{-6pc}{3.5ex plus .1ex minus .2ex}{1.5ex minus .1ex}
\titleformat{\paragraph}[runin]{\bfseries}{\theparagraph}{1em}{}[\hspace{1em}]
\titlespacing{\paragraph}{0pt}{0pt}{0pt}

\usepackage{calc}
\newlength{\indentedpara}


\begin{document}
\setlength{\indentedpara}{\linewidth-2em}
\appendix
\section{参考文献标引体系编制法}
\stepcounter{subsection}
\stepcounter{subsection}
\subsection{著者-出版年制}
\subsubsection{正文中的引用}
\paragraph{} 正文引用的文献采用著者-出版年制时,各篇文献的标注内容由著者姓氏与出版年构成,并置于“()”内。倘若只标注著者姓氏无法识别该人名时,可标注著者姓名,例如中国人、韩国人、日本人用汉字书写的姓名。集体著者著述的文献可标注机关团体名称。倘若正文中已提及著者姓名,则在其后的“()”内只著录出版年。

\begin{refsection}
\textbf{示例:}引用单篇文献

the notion of an invisible college has been explored in thesciences\cite{CRANE1972--}.Its absence among historians was noted by Stieg\yearcite{STIEG1981-549-560} ...

\textbf{参考文献:}

\parbox{2em}{}\parbox{\indentedpara}{\printbibliography[heading=none]}
\end{refsection}

\paragraph{} 正文中引用多著者文献时,对欧美著者只需标注第一个著者的姓,其后附“ et al.”;对于中国著者应标注第一著者的姓名,其后附“等”字。姓氏与“ et al.”“等”之间留适当空隙。

\paragraph{} 在参考文献表中著录同一著者在同一年出版的多篇文献时,出版年后应用小写字母 a, b, c…区别。

\begin{refsection}
\textbf{示例 1:}引用同一著者同年出版的多篇中文文献
\nocite{贾君枝2023编目,贾君枝2023短评}

\parbox{2em}{}\parbox{\indentedpara}{\printbibliography[heading=none]}
\end{refsection}

\begin{refsection}
\textbf{示例 2:}引用同一著者同年出版的多篇英文文献
\nocite{KENNEDY1975-311-386,KENNEDY1975-339-360}

\parbox{2em}{}\parbox{\indentedpara}{\printbibliography[heading=none]}
\end{refsection}


\paragraph{} 多次引用同一著者的同一文献,在正文中标注著者与出版年,并在“()”外以角标的形式著录引文页码。

\begin{refsection}
\textbf{示例:}多次引用同一著者的同一文献

主编靠编辑思想指挥全局已是编辑界的共识\cite{张忠智1997-33-34},然而对编辑思想至今没有一个明确的界定,故不妨提出一个构架……参与讨论。
由于“思想”的内涵是“客观存在反映在人的意识中经过思维活动而产生的结果”
\pagescite[1194]{中国社会科学院语言研究所词典编辑室1996--},所以“编辑思想”的内涵就是编辑实践反映在编辑工作者
的意识中,“经过思维活动而产生的结果”。
……《中国青年》杂志创办人追求的高格调-
理性的成熟与热点的凝聚\cite{刘彻东1998-38-39},表明其读者群的文化的品位的高层次……“方针”指“引导事业前进的方向和目标”
\pagescite[235]{中国社会科学院语言研究所词典编辑室1996--}。
……对编辑方针,1981年中国科协副主席裴丽生曾有过科学的论断—“自然科学学术期刊必须坚持以马列主义、毛泽东思想为指导,贯彻为国民经济发展服务,理论与实践相结合,普及与提高相结合,‘百花齐放,百家争鸣’的方针。” \cite{裴丽生1981-2-10}它完整地回答了为谁服务怎样服务,如何服务得更好的间题。

…………


\textbf{参考文献:}

\newrefcontext[sorting=gb7714-2015]{}
\parbox{2em}{}\parbox{\indentedpara}{\printbibliography[heading=none]}
\end{refsection}


\subsubsection{参考文献表示例}
\begin{refsection}
参考文献表采用著者-出版年制组织时,各篇文献首先按文种集中,可分为中文、日文、西文、俄文、其他文种 5 部分;然后按著者字顺和出版年排列。中文文献可以按著者汉语拼音字顺排列(参见 A.3.1),也可以按著者的笔画笔顺排列。

\nocite{尼葛洛庞帝1996--,汪冰1997-16-16,杨宗英1996-24-29,Baker1995--,Chernik1982--,Dowler1995-5-26}

\textbf{示例:}

\parbox{2em}{}\parbox{\indentedpara}{\printbibliography[heading=none]}
\end{refsection}

\end{document}
