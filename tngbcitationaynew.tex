
% !Mode:: "TeX:UTF-8"
% 用于测试gb7714-2015ay样式,实现GB/T 7714-2015 标准说明中给出的著者年份制示例
\documentclass{article}
\usepackage[10pt]{ctex}
\setmainfont[BoldFont={cmunsx.otf}]{cmunrm.otf}
\setmonofont{cmuntt.otf}
\setsansfont{cmunsx.otf}
\IfFontExistsTF{SourceHanSerifSC-Regular.otf}%\IfFileExists
    {\setCJKmainfont[BoldFont={simhei.ttf}]{SourceHanSerifSC-Regular.otf}
\setCJKsansfont{simhei.ttf}}{}
\usepackage{xcolor}
\usepackage{toolbox}
\usepackage[colorlinks]{hyperref}
\usepackage{lipsum}
\usepackage[paperwidth=21cm,paperheight=3.0cm,top=0cm,bottom=0cm,left=2.5cm,right=2cm]{geometry}%,showframe
\usepackage{xltxtra,mflogo,texnames}
\usepackage[backend=biber,style=gb7714-2025ay,gbnoauthor,gbpunctwidth=mixed,
sortcase=false,
gbpub=true,
sorting=none
%mergedate=none
%,sortlocale=zh__stroke,nosortothers=true,nohashothers=true,
]{biblatex}%sorting=nyt
\renewcommand{\bibfont}{\zihao{-5}}

\addbibresource{example2025.bib}

%\setcounter{secnumdepth}{4}
%\usepackage{titlesec}
%\titleformat{\section}{\color{white}}{\thesection}{1em}{}
%%\titlespacing{\section}{-6pc}{3.5ex plus .1ex minus .2ex}{1.5ex minus .1ex}
%\titleformat{\paragraph}[runin]{\zihao{5}\bfseries}{\theparagraph}{1em}{}[\hspace{1em}]
%\titlespacing{\paragraph}{0pt}{0pt}{0pt}

\usepackage{calc}
\newlength{\indentedpara}
\pagestyle{empty}

\begin{document}
\setlength{\indentedpara}{\linewidth-2em}
\newgeometry{layoutwidth=21cm,layoutheight=3.5cm,top=0cm,bottom=0cm,left=2.5cm,right=2cm}
\begin{refsection}
%\textbf{示例:}引用单篇文献
the notion of an invisible college has been explored in thesciences\cite{CRANE1972--}.Its absence among historians was noted by Stieg\yearcite{STIEG1981-549-560} ...

\textbf{参考文献:}

\parbox{2em}{}\parbox{\indentedpara}{\printbibliography[heading=none]}
\end{refsection}

\eject

\pdfpagewidth=21cm \pdfpageheight=1.7cm
\newgeometry{layoutwidth=21cm,layoutheight=2.3cm,layoutvoffset=-2mm,top=0cm,bottom=0cm,left=2.5cm,right=2cm}
\begin{refsection}
%\textbf{示例 1:}引用同一著者同年出版的多篇中文文献
\nocite{贾君枝2023编目,贾君枝2023短评}
\vspace{-5mm}
\parbox{2em}{}\parbox{\indentedpara}{\printbibliography[heading=none]}
\end{refsection}


\eject

\pdfpagewidth=21cm \pdfpageheight=2.2cm
\newgeometry{layoutwidth=21cm,layoutheight=2.8cm,layoutvoffset=-1.5mm,top=0cm,bottom=0cm,left=2.5cm,right=2cm}
\begin{refsection}
%\textbf{示例 2:}引用同一著者同年出版的多篇英文文献
\nocite{KENNEDY1975-311-386,KENNEDY1975-339-360}

\parbox{2em}{}\parbox{\indentedpara}{\printbibliography[heading=none]}
\end{refsection}


\eject

\pdfpagewidth=21cm \pdfpageheight=9.8cm
\newgeometry{layoutwidth=21cm,layoutheight=10cm,layoutvoffset=1.5mm,top=0cm,bottom=0cm,left=2.5cm,right=2cm}
\begin{refsection}
%\textbf{示例:}多次引用同一著者的同一文献

主编靠编辑思想指挥全局已是编辑界的共识\cite{张忠智1997-33-34},然而对编辑思想至今没有一个明确的界定,故不妨提出一个构架……参与讨论。
由于“思想”的内涵是“客观存在反映在人的意识中经过思维活动而产生的结果”
\pagescite[1194]{中国社会科学院语言研究所词典编辑室1996--},所以“编辑思想”的内涵就是编辑实践反映在编辑工作者
的意识中,“经过思维活动而产生的结果”。
……《中国青年》杂志创办人追求的高格调-
理性的成熟与热点的凝聚\cite{刘彻东1998-38-39},表明其读者群的文化的品位的高层次……“方针”指“引导事业前进的方向和目标”
\pagescite[235]{中国社会科学院语言研究所词典编辑室1996--}。
……对编辑方针,1981年中国科协副主席裴丽生曾有过科学的论断—“自然科学学术期刊必须坚持以马列主义、毛泽东思想为指导,贯彻为国民经济发展服务,理论与实践相结合,普及与提高相结合,‘百花齐放,百家争鸣’的方针。” \cite{裴丽生1981-2-10}它完整地回答了为谁服务怎样服务,如何服务得更好的间题。

…………


\textbf{参考文献:}

\newrefcontext[sorting=gb7714-2015]{}
\parbox{2em}{}\parbox{\indentedpara}{\printbibliography[heading=none]}
\end{refsection}


\eject

\pdfpagewidth=21cm \pdfpageheight=4.9cm
\newgeometry{layoutwidth=21cm,layoutheight=5.4cm,layoutvoffset=-2mm,top=0cm,bottom=0cm,left=2.5cm,right=2cm}
\begin{refsection}
%参考文献表采用著者-出版年制组织时,各篇文献首先按文种集中,可分为中文、日文、西文、俄文、其他文种 5 部分;然后按著者字顺和出版年排列。中文文献可以按著者汉语拼音字顺排列(参见 A.3.1),也可以按著者的笔画笔顺排列。
%\textbf{示例:}

\nocite{尼葛洛庞帝1996--,汪冰1997-16-16,杨宗英1996-24-29,Baker1995--,Chernik1982--,Dowler1995-5-26}

\parbox{2em}{}\parbox{\indentedpara}{\printbibliography[heading=none]}
\end{refsection}

\end{document}
