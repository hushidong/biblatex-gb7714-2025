
% !Mode:: "TeX:UTF-8"
% 用于测试gb7714-2025样式
\documentclass{article}
\usepackage[fontset=windows]{ctex}
\usepackage[colorlinks,CJKbookmarks,bookmarksnumbered=true]{hyperref}
\usepackage[a4paper,top=3cm,bottom=2cm,left=2.54cm,right=2.54cm]{geometry}

\usepackage[backend=biber,style=gb7714,gbnamefmt=fullname,%casechanger=latex2e,%gbpunctwidth=mixed,
gbfootbib=true,gbalign=gb7714]{biblatex}%sorting=nyt
%\setlength{\bibitemsep}{1pt}
%\setlength{\bibnamesep}{0ex}
%\setlength{\bibinitsep}{0ex}
%\setlength{\bibparsep}{0ex}
\renewcommand{\bibfont}{\zihao{-5}}
\renewcommand\mktitlecase[1]{\MakeSentenceCase{#1}}
%可用\MakeUppercase,\MakeLowercase,\MakeSentenceCase
%\renewcommand\mkbooktitlecase[1]{\MakeLowercase{#1}}%用于调整booktitlecase
%\renewcommand\mkjournaltitlecase[1]{\MakeUppercase{#1}}%用于调整journaltitlecase

\addbibresource{example2025.bib}

\begin{document}

\section{参考文献标引体系-顺序编码制}

\begin{refsection}
示例1:

所谓移情,就是“说话人将自己认同于......他用句子所描述的时间或状态中的一个参与者”
\cite{Sunstein1996-903-903}。《汉语大词典》和张相
\cite{Morri2010--}都认为“可”是“痊愈”,
候精一认为是“减轻”\cite{罗杰斯2011-15-16}。......另外,根据候精一,表示病痛程度减轻的形容词“可”和表示逆转否定的副词“可”
是兼类词\cite{陈登原2000-29-29},这也说明二者应该存在着源流关系。

裴伟提出\cite{Humphrey1971--,KENNEDY1975-311-386}......

莫拉德对稳定区的研究
\cite{KENNEDY1975-311-386,CRANE1972--,Weinstein1974-745-772}......

示例2:

所谓移情,就是“说话人将自己认同于......他用句子所描述的时间或状态中的一个参与者”
\footfullcite{Sunstein1996-903-903}。《汉语大词典》和张相
\footfullcite{Morri2010--}都认为“可”是“痊愈”,
候精一认为是“减轻”\footfullcite{罗杰斯2011-15-16}。......另外,根据候精一,表示病痛程度减轻的形容词“可”和表示逆转否定的副词“可”
是兼类词\footfullcite{陈登原2000-29-29},这也说明二者应该存在着源流关系。

\printbibliography[heading=subbibliography,title={参考文献}]
\end{refsection}

\end{document}
