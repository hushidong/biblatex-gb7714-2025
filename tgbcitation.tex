
% !Mode:: "TeX:UTF-8"
% 用于测试gb7714-2015样式,实现GB/T 7714-2015 标准说明中给出的顺序编码制示例
%测试实现GB/T 7714-2015 标准2-10节给出的顺序编码制示例
\documentclass{article}
\usepackage[fontset=windows,10pt]{ctex}
\usepackage[russian,french,english]{babel}
\setmainfont[BoldFont={cmunsx.otf}]{cmunrm.otf}
\setmonofont{cmuntt.otf}
\setsansfont{cmunsx.otf}
\IfFontExistsTF{SourceHanSerifSC-Regular.otf}%\IfFileExists
    {\setCJKmainfont[BoldFont={simhei.ttf}]{SourceHanSerifSC-Regular.otf}
\setCJKsansfont{simhei.ttf}}{}
%\IfFontExistsTF{font name}{true branch}{false branch}
\usepackage{xcolor}
\usepackage{toolbox}
\usepackage[colorlinks,CJKbookmarks,bookmarksnumbered=true]{hyperref}
\usepackage{lipsum}
\usepackage[a4paper,top=3cm,bottom=2.5cm,left=2.5cm,right=2cm]{geometry}
\usepackage{tcolorbox}

\setcounter{secnumdepth}{4}
\usepackage[nostruts]{titlesec}
\titleformat{\section}{\zihao{5}\bfseries}{\thesection}{1em}{}
\titlespacing{\section}{0pt}{1.5ex plus 1pt minus 1pt}{1.5ex plus 1pt minus 1pt}
\titleformat{\paragraph}[runin]{\zihao{5}\bfseries}{\theparagraph}{1em}{}
%\titlespacing{\paragraph}{5pc}{1.5ex minus .1 ex}{1pc}



\usepackage{xltxtra,mflogo,texnames}
%%注意:这里使用other*来使得babel切换环境时忽略空白,使得条目集中各文献间的标点正确
\usepackage[backend=biber,style=gb7714-2025,gbpub=true,gbpunctwidth=mixed,
gbfootbib=true,gbalign=gb7714-2015,autolang=other*]{biblatex}%sorting=nyt
\setlength{\bibitemsep}{1pt}
\setlength{\bibnamesep}{0ex}
\setlength{\bibinitsep}{0ex}
\setlength{\bibparsep}{0ex}
\renewcommand{\bibfont}{\zihao{-5}}



% printbiblist命令需要一个与其参数同名的文献输出驱动,详见92-bibliographylists.tex
\DeclareBibliographyDriver{authorexample}{%
%\usebibmacro{begentry}%
\begingroup
\usebibmacro{author/editor+others/translator+others}\par
%\usebibmacro{finentry}
\endgroup
}


\DeclareFieldFormat{titlea}{#1}

\renewbibmacro*{titlea}{%
  \ifboolexpr{%
    test{\iffieldundef{title}}%
    and
    test{\iffieldundef{subtitle}}%
  }%
    {}%
    {\printtext[titlea]{\bibtitlefont%增加字体控制命令
       \printfield[titlecase]{title}%
       \ifboolexpr{test {\iffieldundef{subtitle}}}%这里增加了对子标题的判断,解决不判断多一个点的问题
       {}{\setunit{\subtitlepunct}%
       \printfield[titlecase]{subtitle}}%
       \iffieldundef{titleaddon}{}%判断一下titleaddon,否则直接加可能多一个空格
        {\setunit{\subtitlepunct}\printfield{titleaddon}}%
}}}


\DeclareFieldFormat{titleb}{#1}
\renewbibmacro*{titleb}{%
\ifboolexpr{%
    test{\iffieldundef{title}}%
    and
    test{\iffieldundef{subtitle}}}%
    {}%
    {\printtext[titleb]{\bibtitlefont%增加字体控制命令
       \printfield[titlecase]{title}%
       \ifboolexpr{test {\iffieldundef{subtitle}}}%增加了对子标题的判断
            {}{\setunit{\subtitlepunct}\printfield[titlecase]{subtitle}}%
       \iffieldundef{titleaddon}{}{\setunit{\subtitlepunct}\printfield{titleaddon}}%判断一下titleaddon
       \ifboolexpr{test{\ifentrytype{book}} and (not test{\iffieldundef{volume}})}%判断一下book类有没有volume
            {\setunit{\subtitlepunct}\printfield{volume}}{}%
       \ifboolexpr{(test{\ifentrytype{archive}} or  test{\ifentrytype{online}}) and (not test{\iffieldundef{number}})}%判断一下archive类有没有number
            {\setunit{\subtitlepunct}\printfield{number}}{}%
       \iftoggle{bbx:gbtype}%
            {\iffieldundef{entrysubtype}{\printfield[gbtypeflag]{usera}}%在标题后直接给出文献标识字母,判断一下,是否是报纸和标准
            {\iffieldequalstr{entrysubtype}{standard}{\printfield[gbtypeflags]{usera}}%判断是否为标准
                  {\iffieldequalstr{entrysubtype}{news}{\printfield[gbtypeflagn]{usera}}% 判断是否为报纸
                  {\printfield[gbtypeflag]{usera}}}}}{}%%其它
       }%
    }%
}

\DeclareBibliographyDriver{titlexamplea}{%
\usebibmacro{begentry}%
\iffieldundef{title}{\usebibmacro{journal}}{\usebibmacro{titlea}}
\renewcommand*{\finentrypunct}{}%
\usebibmacro{finentry}
}


\DeclareBibliographyDriver{titlexample}{%
\usebibmacro{begentry}%
\iffieldundef{title}{\usebibmacro{journal+issuetitle}}{\usebibmacro{titleb}}%
\renewcommand*{\finentrypunct}{}%
\usebibmacro{finentry}
}

\DeclareBibliographyDriver{editionexample}{%
%\usebibmacro{begentry}%
\begingroup
\printfield{edition}\par
%\usebibmacro{finentry}
\endgroup
}

\DeclareBibliographyDriver{pubexample}{%
\usebibmacro{begentry}%
\ifboolexpr{test {\ifentrytype{online}} or test {\ifentrytype{archive}}}%
{\usebibmacro{institution+location+date}}{\usebibmacro{publisher+location+date}}%\par
\usebibmacro{chapter+pages}%
\usebibmacro{doi+eprint+url}%从下面移动到上面来,因为gbt2015的url需直接放在页码后面。
\renewcommand*{\finentrypunct}{}%
\usebibmacro{finentry}
}

\DeclareBibliographyDriver{dateexample}{%
\usebibmacro{begentry}%
\ifentrytype{online}{\usebibmacro{modifydate}\usebibmacro{online:url+urldate}}{%
\ifboolexpr{test {\ifkeyword{news}} or test {\ifentrytype{patent}}}
    {\usebibmacro{newsdate}}
    {\usebibmacro{institution+location+date}}}%\par
\usebibmacro{chapter+pages}%
\usebibmacro{doi+eprint+url}%从下面移动到上面来,因为gbt2015的url需直接放在页码后面。
\renewcommand*{\finentrypunct}{}%
\usebibmacro{finentry}%
}

\DeclareBibliographyDriver{dvnpexample}{%
\usebibmacro{issue+date}%
\iffieldundef{volume}{}{\setunit{\jourdatevoldelim}}%
\usebibmacro{volume+number+eid}%把卷期放到年份后面
\usebibmacro{note+pages}\par%
}


\usepackage{calc}

\setlength{\biblabelsep}{0.5em}
\defbibenvironment{exampleenv}
  {\list
     {\textbf{示例\printfield{labelnumber}:}}
     {\setlength{\topsep}{0ex}%
      \setlength{\partopsep}{0ex}%
      \setlength{\parskip}{0ex}%
      \setlength{\labelwidth}{4em}%
      \setlength{\leftmargin}{\labelwidth+20pt}%
      \setlength{\labelsep}{\biblabelsep}%
      \addtolength{\leftmargin}{\labelsep}%
      \setlength{\itemsep}{\bibitemsep}%
      \setlength{\parsep}{\bibparsep}%
      \renewcommand*{\makelabel}[1]{##1\hss}}}
  {\endlist}
  {\item}

\defbibenvironment{indentegenv}
  {\list
     {[\printfield{labelnumber}]}
     {\setlength{\topsep}{0ex}%
      \setlength{\partopsep}{0ex}%
      \setlength{\parskip}{0ex}%
     \setlength{\labelwidth}{1.5em}%
      \setlength{\leftmargin}{\labelwidth+20pt}%
      \setlength{\labelsep}{\biblabelsep}%
      \addtolength{\leftmargin}{\labelsep}%
      \setlength{\itemsep}{\bibitemsep}%
      \setlength{\parsep}{\bibparsep}%
      \renewcommand*{\makelabel}[1]{##1\hss}}}
  {\endlist}
  {\item}


\defbibheading{biblist}[\bibname]{%
\paragraph*{#1}}

\addbibresource{example2025.bib}

\title{信息与文献 参考文献著录规则\footnote{说明:本文档为GB/T 7714-2025的简单复现版本,作为GB/T 7714-2025 中的著录标准和顺序编码制示例,用于gb7714-2025样式更新后与GB/T 7714-2025标准文档的内容进行比较,以确定更新没有引入样式方面的BUG,对比文档为stdgbT7714-2025.pdf,该文档经过多次比较,明确与GB/T 7714-2025中的示例完全一致。}}
\author{}
\date{}

\begin{document}

\maketitle
\vspace{-1cm}%
\titleformat{\subsection}[display]{\zihao{5}\bfseries}
{\thesubsection}{1em}{\vspace{-2ex}\hspace{2em}}
\titlespacing{\subsection}{0pt}{2pt plus 1pt minus 1pt}{2pt plus 1pt minus 1pt}


\section{范围}

本文件规定了参考文献的著录用文字、著录用符号、著录通则、各文献类型的著录项目与著录格式
以及参考文献标引体系编制法。
本文件适用于著者和编辑著录参考文献。
本文件不适用于图书馆员、文献目录编制者以及索引编辑者作为文献著录规则使用。

\section{规范性引用文件}

下列文件中的内容通过文中的规范性引用而构成本文件必不可少的条款。其中,注日期的引用文件,
仅该日期对应的版本适用于本文件;不注日期的引用文件,其最新版本(包括所有的修改版)适用于本文件。

GB/T 4894—2024 信息与文献 基础与术语

GB/T 7408 日期和时间 信息交换表示法

GB/T 28039 中国人名汉语拼音字母拼写规则

ISO 4 信息与文献 出版物题名和标题缩写规则( Information and documentation—Rules for the
abbreviation of title words and titles of publications)

\section{术语和定义}

GB/T 4894—2024界定的以及下列术语和定义适用于本文件。

\subsection{参考文献 reference}
对信息资源或其中一部分进行准确和详细著录的数据,位于文末或文中,以便识别、定位和检索相
关信息。

\subsection{电子资源 electronic resource}
以数字方式将图、文、声、像等信息存储在磁、光、电介质上,通过计算机、网络或相关设备使用
的记录有知识内容或艺术内容的信息资源。

\subsection{合订题名 title of the individual works}
由两种或两种以上的作品汇编而成的无总题名的文献中各部著作的题名。

\subsection{会议录 proceedings; transactions}
包含向大会提交的论文、通常还有论文的讨论和与论文相关事务等内容报道的文献。
[来源: GB/T 4894-2024, 3.4.7.34]

\subsection{连续出版物 serial; serial publication}
以连续的各个部分发行的、通常具有数字或年代标识的、计划无限期发行,不论周期长短的印刷文
献或非印刷形式的出版物。
注:包括:期刊、报纸等。

[来源: GB/T 4894-2024, 3.4.1.28.1]

\subsection{数据集 dataset}
相似或相关数据合乎逻辑的集合或分组,通常用作记录或用于研究。

\subsection{顺序编码制 numeric system}
一种引文参考文献( 3.9)的标注体系,即引文采用序号标注,参考文献表按引文的序号排序。

\subsection{析出文献 component part}
由某个责任者( 3.13)提供的、构成可能涉及多个责任者( 3.13)的主文献一部分的文献。
注:析出文献可以是连续出版物( 3.5)中的论文,图书中具有独立作者、独立篇名的文献等。

\subsection{引文参考文献 cited reference}
著者为撰写或编辑作品而引用的信息资源。

\subsection{永久标识符 persistent identifier; PID}
数字对象的唯一标识符,通过独立于其物理位置或当前所有权提供访问,确保对其的永久存取。
注:永久标识符包括数字对象标识符( DOI)、统一资源名称( URN)等。

[来源: GB/T 4894-2024, 3.2.5.25,有修改]

\subsection{预印本 preprint}
尚未通过同行评审但仍可广泛使用的文稿。


\subsection{阅读型参考文献 reading reference}
著者为撰写或编辑作品而阅读过的信息资源,或供读者进一步阅读的信息资源。

\subsection{责任者 creator}
在开展个人或团体活动时负责创造、积累和/或保管信息资源的任何实体(法人团体、家族或个人) 。


\subsection{著者-出版年制 name and data system}
一种引文参考文献( 3.9)的标注体系,即引文采用著者-出版年标注,参考文献表按著者字顺和出
版年排序。


\titleformat{\subsection}[runin]{\zihao{5}\bfseries}{\thesubsection}{1em}{\hspace{1em}}
\section{著录信息源}

参考文献的著录信息源是被著录的信息资源本身。图书、连续出版物、会议录、学位论文、报告、
标准、专利、网站和网页、档案、地图、数据集、预印本可依据题名页、版权页、封面等主要信息源著录各个著录项目;图书、连续出版物中的析出文献依据参考文献本身著录析出文献的信息,并依据主要信息源著录析出文献的出处;电子资源依据其存储介质上提供的主要信息源著录各个著录项目。

\section{著录用文字}
\subsection{} 参考文献原则上要求用信息资源本身的语种著录。必要时,可采用双语著录。用双语著录参考文
献时,首先应用信息资源的原语种著录,然后用其他语种著录。

\textbf{示例1}:用原语种著录参考文献

\begin{refsection}
\nocite{周鲁卫2011--}
\nocite{japaneserefc}
\nocite{RUDDOCK2009--}
\nocite{russianrefc}

{\printbibliography[env=indentegenv,heading=none]}
\end{refsection}

\textbf{示例2:}用韩中两种语种著录参考文献
\begin{refsection}
\defbibentryset{李炳穆set}{kereanrefb,李炳穆2005--}
\defbibentryset{图书馆信息政策set}{kereanrefc,图书馆信息政策委员会2007--}
\nocite{李炳穆set}
\nocite{图书馆信息政策set}


{\printbibliography[heading=none,env=indentegenv]}
\end{refsection}

\textbf{示例3:}用中英两种语种著录参考文献
\begin{refsection}
\defbibentryset{熊平set}{熊平2005--,xiong2005--}
\defbibentryset{脱贫攻坚2021set}{脱贫攻坚2021,脱贫攻坚2021en}
\nocite{熊平set}
\nocite{脱贫攻坚2021set}

{\printbibliography[heading=none,env=indentegenv]}
\end{refsection}


\subsection{} 著录数字时,应保持信息资源原有的形式。但是,卷期号、页码、出版年、创建或修改日期、引用日期、顺序编码制的参考文献序号等宜用阿拉伯数字著录,其中日期著录应遵循 GB/T 7408 中的有关
规定。

\subsection{} 使用汉语拼音著录中国人名,按照GB/T 28039的有关规定拼写。

\subsection{} 作为集体责任者的机关团体名称(见7.1.4, A.3.1.1)、出版项中附在出版地之后的省名、州名、国名等(见7.5.2.1)以及出版者(见7.5.3.1)可按国际公认的方法缩写。

\subsection{} 西文期刊刊名的缩写可参照 ISO 4 的规定。

\subsection{} 著录西文文献时,大写字母的使用应符合信息资源本身文种的习惯用法。

\section{著录用符号}

参考文献使用下列规定的标识符号:

. 用于题名项、析出文献题名项、其他责任者、析出文献其他责任者、连续出版物的“年卷期或其
他标识”项、版本项、出版项、地图比例尺、地图尺寸、连续出版物中析出文献的出处项、获取
和访问路径以及永久标识符前。每一条参考文献的结尾可用“.”号标识。

: 用于其他题名信息、报告编号、标准号、专利申请号、档号、出版者、引文页码、析出文献的页
码前。

, 用于同一著作方式的责任者、“等”“译”字样、出版年、期刊年卷期标识中的年和卷号前。

; 用于同一责任者的合订题名以及期刊后续的年卷期标识与页码前。

// 用于析出文献的出处项前。

( ) 用于期刊年卷期标识中的期号、报纸的版次、电子资源的创建或修改日期以及非公元纪年的出
版年。

[ ] 用于文献序号、文献类型和文献载体标识、电子资源的引用日期以及自拟的信息。

/ 用于合期的期号间以及文献载体标识前。

- 用于起讫序号和起讫页码间。

注: “.”“[ ]”用半角符号,其他标识符号用全角符号。\footnote{为方便考虑,在实现上考虑三种实现,一种是全部半角,一种是全部全角,一种是全角半角混合。当前标准对应的就是最后这一种。三种方式通过选项 gbpunctwidth 来切换。gbpunctwidth=half,full,mixed 分别对应这三种方式。}

%\begin{tcolorbox}[colback=yellow!80,boxrule=0pt]
%注意:标点符号要按照全半角、部分全角、全全角的方式以选项方式设置。
%\end{tcolorbox}



\titleformat{\subsection}[hang]{\zihao{5}\bfseries}{\thesubsection}{1em}{}
\titleformat{\subsubsection}[runin]{\zihao{5}\bfseries}{\thesubsubsection}{1em}{}%\hspace{1em}
\titlespacing{\subsubsection}{0pt}{2pt plus 1pt minus 1pt}{5pt plus 1pt minus 1pt}
\section{著录规则}
\subsection{主要责任者或其他责任者}
\subsubsection{} 个人责任者采用姓在前名在后的著录形式。欧美责任者的名可用缩写字母,缩写名后省略缩写点。
欧美责任者的中译名只著录其姓;同姓不同名的欧美责任者,其中译名不仅要著录其姓,还应著录其名的首字母。 用西文或汉语拼音字母著录个人作者, 姓应全部著录,字母全大写,名可缩写为首字母;如用首字母无法识别该人名时,可著录全名\footnote{注意:下面示例中“昂温”文献的情况,姓名是中英文混合的情况是比较特殊的,因为存在中文,所以判定为中文,但为了输出的形式保持英文中姓和名之间的空格,因此bib文件中的输入以\{\}进行保护。特别注意:示例3中的结果是因为处理中文情况:“姓, 名”时,会去掉逗号合并姓和名导致的。}。

\begin{refsection}

\nocite{李时珍--}
\nocite{乔纳斯--}
\nocite{昂温1988--}
\nocite{GPS1988--}
\nocite{丸山敏秋--}
\nocite{凯西尔--}
\nocite{Einstein--}
\nocite{Williams-ellis--}
\nocite{morgan--}
\nocite{lijianning--a}
\nocite{lijianning--b}

{\printbiblist[heading=none,env=exampleenv]{authorexample}}
\end{refsection}

\subsubsection{}  著作方式相同的责任者不超过 3 个时,全部照录。超过 3 个时,著录前 3 个责任者,其后加“,等”或与之相应的词。
\begin{refsection}
\nocite{钱学森--}
\nocite{李四光--}
\nocite{印森林--}
\nocite{fordham--}

{\printbiblist[heading=none,env=exampleenv]{authorexample}}
\end{refsection}

\subsubsection{} 无责任者或者责任者情况不明的文献,“主要责任者”项应注明“佚名”或与之相应的词。凡采用顺序编码制(见 A.2)组织的参考文献可省略此项,直接著录题名。\footnote{注意:佚名的处理是全局的,也就是一个文档统一的,所以这里的示例与整个文档是一致的,而不是单独局部化处理的,所以没有给出佚名,若要给出则可以设置选项 gbnoauthor=true。}

\begin{refsection}
\nocite{帛画2023,anon1981-628}

{\printbibliography[heading=none,env=indentegenv]}
\end{refsection}

\subsubsection{} 凡是对文献负责的机关团体名称,通常根据著录信息源著录。机关团体名称应由上至下分级著录,上下级间用“ .”分隔,用汉字书写的机关团体名称除外。

\begin{refsection}
\nocite{中国科学院物理研究所--}
\nocite{贵州省土穰普查办公室--}
\nocite{AmericanChemicalSociety--}
\nocite{StanfordUniversity--}

{\printbiblist[heading=none, env=exampleenv]{authorexample}}
\end{refsection}

\subsection{题名}

\subsubsection{} 题名按著录信息源所载的内容著录。如无题名信息,可根据信息源所载内容自拟题名。
\begin{refsection}
\nocite{西游记词汇--}
\nocite{张子正蒙注--}
\nocite{化学动力学和反应器原理--}
\nocite{袖珍神学--}
\nocite{北京师范大学学报--}
\nocite{Gasesinsea--}
\nocite{jmath--}


{\printbiblist[heading=none, env=exampleenv]{titlexamplea}}
\end{refsection}

\subsubsection{} 同一责任者的多个合订题名,著录前 3 个合订题名。对于不同责任者的多个合订题名,可以只著
录第一个或处于显要位置的合订题名。在参考文献中不著录并列题名。

\begin{refsection}
\nocite{为人民服务--}
\nocite{大趋势--}

{\printbiblist[heading=none, env=exampleenv]{titlexamplea}}
\end{refsection}

\subsubsection{} 其他题名信息根据信息资源外部特征的具体情况决定取舍。其他题名信息包括副题名,说明题名文字,多卷书的分卷书名、卷次、册次等。

\begin{refsection}
\nocite{地壳运动--}
\nocite{三松堂--}
\nocite{世界出版业--}
\nocite{ECL集成电路--}
\nocite{中国科学技术史--}
\nocite{中国科学--}
\nocite{AsianPacificjournal--}

{\printbiblist[heading=none, env=exampleenv]{titlexamplea}}
\end{refsection}


\subsection{文献类型和载体标识符}

应按照附录 B 著录。电子资源既应著录文献类型标识,也应著录文献
载体标识。

\begin{refsection}
\nocite{马寅初讲义--}
\nocite{商鞅战秋菊--}
\nocite{严复思想--}
\nocite{中子反射--}
\nocite{信息与文献--}
\nocite{智能戒指--}
\nocite{Quantumfield--}

{\printbiblist[heading=none, env=exampleenv]{titlexample}}
\end{refsection}

\subsection{版本}

印刷版第 1 版不著录,其他版本说明应著录。版本宜用阿拉伯数字、序数缩写形式或其他标识表
示。古籍的版本应按原文客观照录,如“写本”“抄本”“刻本”“活字本”等。


\begin{refsection}
\nocite{egbookeda--}
\nocite{egbookedb--}
\nocite{egbookedc--,egbookedf--}
\nocite{egbookedd--}
\nocite{egbookede--}
{\printbiblist[heading=none, env=exampleenv]{editionexample}}
\end{refsection}

\subsection{出版项}

\subsubsection{通用要求}

出版项应按出版地、出版者、出版日期顺序著录。

本文件中的以下项目也应按照出版项的要求著录:

——会议录中的会议地点、会议名称、会议年份(见8.6);

——学位论文中的学位授予单位所在地、学位论文授予单位、学位授予年(见8.7);

——档案中的收藏地、收藏者、档案形成日期(见8.12);

——报告发布日期(见8.8);

——专利公告(公开)日期(见8.10);

——数据集中的创建机构(见8.14);

——预印本中的出版平台(见8.15)。

\begin{refsection}
\nocite{egbookpuba--}
\nocite{egbookpubb--}
\nocite{egbookpubc--}
\nocite{egbookpubd--}
\nocite{egbookpube--}
{\printbiblist[heading=none, env=exampleenv]{pubexample}}
\end{refsection}

\subsubsection{出版地}

\paragraph{} 出版地应著录出版者所在地的城市名称。对同名异地或不为人们熟悉的城市名,宜在城市名后附省、州名或国名等限定语。

\begin{refsection}
\nocite{egbookpubaddressa--}
\nocite{egbookpubaddressb--}


{\printbiblist[heading=none, env=exampleenv]{pubexample}}
\end{refsection}


\paragraph{} 文献中载有多个出版地,宜只著录第一个或处于显要位置的出版地。

\begin{refsection}
\nocite{egbookpubaddressc--}
\nocite{egbookpubaddressd--}


{\printbiblist[heading=none, env=exampleenv]{pubexample}}
\end{refsection}

\paragraph{} 无出版地的中文文献著录“出版地不详”,外文文献著录“ S.l.”,并置于方括号内。无出版地的电子资源可省略此项。

\begin{refsection}

\nocite{egbookpubaddresse--}
\nocite{egbookpubaddressf--}
\nocite{egbookpubaddressg--}

{\printbiblist[heading=none, env=exampleenv]{pubexample}}
\end{refsection}


\subsubsection{出版者}

\paragraph{} 出版者可以按著录信息源所载的形式著录,也可以按国际公认的简化形式或缩写形式著录。

\begin{refsection}
\nocite{egbookpubpublishera--}
\nocite{egbookpubpublisherb--}
\nocite{egbookpubpublisherc--}

{\printbiblist[heading=none, env=exampleenv]{pubexample}}
\end{refsection}

\paragraph{} 文献中载有多个出版者,宜只著录第一个或处于显要位置的出版者。


\begin{refsection}
\nocite{egbookpubpublisherd--}

{\printbiblist[heading=none, env=exampleenv]{pubexample}}
\end{refsection}

\paragraph{} 无出版者的中文文献著录“出版者不详”,外文文献著录“ s.n.”,并置于方括号内。无出版者的电子资源可省略此项。

\begin{refsection}

\nocite{egbookpubpublishere--}
\nocite{egbookpubpublisherf--}

{\printbiblist[heading=none, env=exampleenv]{pubexample}}
\end{refsection}



\subsubsection{出版日期}

\paragraph{} 出版年采用公元纪年,并用阿拉伯数字著录。如有其他纪年形式时,将原有的纪年形式置于“(\space  )”内。

\begin{refsection}
\nocite{egbookpubdatea--}
\nocite{egbookpubdateb--}

{\printbiblist[heading=none, env=exampleenv]{dateexample}}
\end{refsection}

\paragraph{} 报纸的出版日期按照“ YYYY-MM-DD”格式,用阿拉伯数字著录。


\begin{refsection}
\nocite{egbookpubdatec--}

{\printbiblist[heading=none, env=exampleenv]{dateexample}}
\end{refsection}

\paragraph{} 专利文献的公告日期或公开日期按照“ YYYY-MM-DD”格式,用阿拉伯数字著录。

\begin{refsection}
\nocite{egbookpubdateh--}

{\printbiblist[heading=none, env=exampleenv]{dateexample}}
\end{refsection}

\paragraph{} 出版年无法确定时,可依次选用版权年、印刷年、估计的出版年。估计的出版年应置于方括号
内。

\begin{refsection}

\nocite{egbookpubdated--}
\nocite{egbookpubdatee--}
\nocite{egbookpubdatef--}

{\printbiblist[heading=none, env=exampleenv]{dateexample}}
\end{refsection}




\subsection{创建或修改日期、引用日期}

电子资源的创建或修改日期、引用日期按照“ YYYY-MM-DD”格式,用阿拉伯数字著录。
\begin{refsection}
\nocite{egbookpubdateg--}

{\printbiblist[heading=none, env=exampleenv]{dateexample}}
\end{refsection}

\subsection{页码}
析出文献的页码或引文页码,应采用阿拉伯数字著录(参见8.5、 A.2.1.3、 A.3.1.4)。引自序言
或扉页题词的页码,可按实际情况著录。无页码有文章编号的,应将文章编号按页码著录。

\begin{refsection}
\nocite{曹凌2011-19-}
\nocite{钱学森2001--}
\nocite{冯友兰2008--}
\nocite{李约瑟1991--}
\nocite{DUBAR2013--}
\nocite{MAURYA2023}

{\printbibliography[heading=none,env=exampleenv]}
\end{refsection}


\subsection{获取和访问路径}
根据电子资源在互联网中的实际情况,著录其获取和访问路径。

\begin{refsection}

\nocite{储大同2010-721-724,weiner2010-38}

{\printbibliography[heading=none,env=exampleenv]}
\end{refsection}

\subsection{永久识别符}
获取和访问路径中含永久标识符时,可不著录永久标识符。当不含永久标识符时,可按照原文如实著
录永久标识符。

\begin{refsection}

\nocite{刘乃安2000-17-18,Deverell2013-21-22,GROSS2016LM}

{\printbibliography[heading=none,env=exampleenv]}
\end{refsection}


\titleformat{\subsubsection}[hang]{\zihao{5}\bfseries}{\thesubsubsection}{1em}{}%\hspace{1em}
\section{各文献类型著录项目与著录格式}
\subsection{概述}

本文件分别规定了图书、图书中的析出文献、连续出版物、连续出版物中的析出文献、会议录、学
位论文、报告、标准、专利、网站和网页、档案、地图、数据集、预印本的著录项目和著录格式。著录项目(元素)设必备状态与可选状态。

\subsection{图书}

\subsubsection{著录项目}

主要责任者:有则必备;
题名;必备;
其他题名信息:有则必备;
文献类型标识:必备;
文献载体标识:电子资源必备;
其他责任者:可选;
版本:有则必备;
出版地:有则必备;
出版者:有则必备;
出版年:有则必备;
引文页码:有则必备;
获取和访问路径:电子资源必备;
永久标识符:电子资源可选。

\subsubsection{著录格式}

主要责任者. 题名:其他题名信息[文献类型标识/文献载体标识]. 其他责任者. 版本. 出版地:
出版者,出版年:引文页码. 获取和访问路径. 永久标识符.

\begin{refsection}

\nocite{张伯伟2002--,
陈登原2000-29-29,
王夫之1865--,
顾炎武1992--,
1962-50-50,
战德臣2019,
哈里森沃尔德伦2012-235-236,
牛永敢2019,
美国妇产科医师学会2010-38-39,
中国企业投资协会2013--,
赵学功2001--,
中国造纸学会2003--,
PEEBLES2001--,
SADOCK2009,
InstituteForArt2023,
Kinchy2012-50-50,
Praetzellis2011-13-13}

\textbf{示例:}

{\printbibliography[heading=none,env=indentegenv]}
\end{refsection}


\subsection{图书中的析出文献}

\subsubsection{著录项目}

析出文献主要责任者 有则必备;
析出文献题名 必备;
析出文献其他题名信息 有则必备;
文献类型标识 必备;
文献载体标识 电子资源必备;
析出文献其他责任者 可选;
图书主要责任者 有则必备;
图书题名 必备;
其他题名信息 有则必备;
版本 有则必备;
出版地 有则必备;
出版者 有则必备;
出版年 有则必备;
引文页码 有则必备;
获取和访问路径 电子资源必备;
永久标识符 电子资源可选。

\subsubsection{著录格式}

析出文献主要责任者. 析出文献题名:其他题名信息[文献类型标识/文献载体标识]. 析出文献其
他责任者//图书主要责任者. 图书题名:其他题名信息. 版本. 出版地:出版者,出版年:引文页码.
获取和访问路径. 永久标识符.\footnote{与之前版本的主要区别是增加了析出文献其他责任者的信息,这里的其他责任者通常只是译者而不会是编者,因为编者也可能是图书的主要责任者。}

\begin{refsection}

\nocite{
1988-590-590,
阿扬2023,
王夫之2011-1109-1109,
程根伟1999-32-36,
陈晋镳1980-56-114a,
马克思2013-302-302,
楼梦麟2011-11-12,
Weinstein1974-745-772,
Roberson2011-1-36
}


\textbf{示例:}

{\printbibliography[heading=none,env=indentegenv]}
\end{refsection}

\subsection{连续出版物}

\subsubsection{著录项目}

主要责任者 有则必备;
题名 必备;
其他题名信息 有则必备;
文献类型标识 必备;
文献载体标识 电子资源必备;
年卷期或其他标识 有则必备;
出版地 有则必备;
出版者 有则必备;
出版年 有则必备;
获取和访问路径 电子资源必备;
永久标识符 电子资源可选。

\subsubsection{著录格式}

主要责任者. 题名:其他题名信息[文献类型标识/文献载体标识]. 年,卷(期) -年,卷(期) .
出版地:出版者,出版年. 获取和访问路径. 永久标识符.



\begin{refsection}

\nocite{
中华医学会湖北分会1984----,
中国图书馆学会1957--1990--,
AAAS1883----,
Publiclibrary1979}


\textbf{示例:}

{\printbibliography[heading=none,env=indentegenv]}
\end{refsection}

\subsection{连续出版物中的析出文献}

\subsubsection{通用要求}

凡是从期刊中析出的文章,应在刊名之后注明其年、卷、期、页码等。阅读型参考文献的页码著录
文章的起讫页或起始页,引文参考文献的页码著录引用信息所在页。

\begin{refsection}

\nocite{egdatevolnumpagea--,egdatevolnumpageb--,%
egdatevolnumpagec--,egdatevolnumpaged--,%
egdatevolnumpagee--}

{\printbiblist[heading=none,env=exampleenv]{dvnpexample}}
\end{refsection}

凡是在同一期刊上连载的文献,其后续部分不必另行著录,可在原参考文献后直接注明后续部分的
年、卷、期、页码等。\footnote{这里这种复杂的年卷期形式不便于自动化处理,所以可以直接在year域中给出需要复杂信息形式,下面的示例中给出了半角符号和全局符号下的两种示例。}

\begin{refsection}

\nocite{egdatevolnumpagef--,egdatevolnumpagefull--}

{\printbiblist[heading=none,env=exampleenv]{dvnpexample}}
\end{refsection}


凡是从报纸中析出的文献,应在报纸名后著录其出版日期与版次。


\begin{refsection}

\nocite{egdatevolnumpageg--}

{\printbiblist[heading=none,env=exampleenv]{dvnpexample}}
\end{refsection}


\subsubsection{著录项目}

析出文献主要责任者 有则必备;
析出文献题名 必备;
析出文献其他题名信息 有则必备;
文献类型标识 必备;
文献载体标识 电子资源必备;
连续出版物题名 有则必备;
其他题名信息 有则必备;
年卷期标识与页码等 有则必备;
获取和访问路径 电子资源必备;
永久标识符 电子资源可选。

\subsubsection{著录格式}

析出文献主要责任者. 析出文献题名[文献类型标识/文献载体标识]. 连续出版物题名:其他题名
信息,年,卷(期):页码. 获取和访问路径. 永久标识符.\footnote{相比前版的关键差别是不输出urldate了,也没有修改或更新日期,同时示例中给出一个在线文档的要求,要求给出详细日期。因为其他文献也可能带有网址,所以只用网址无法区分。因此考虑利用url存在且卷期都不存在这一条件进行区分,比如示例中的第8条文献。}

\begin{refsection}

%\begin{tcolorbox}[colback=yellow!80,boxrule=0pt]
%注意:在线的时间是完整日期,如何区分很重要
%\end{tcolorbox}


\nocite{杨洪升2013-56-75}
\nocite{丁文祥2000--}
\nocite{于潇2012-1518-1523}
\nocite{李炳穆2008-6-12}
\nocite{陈建军2010-93-93,陈缮真2022--,李幼平2010-225-228}
\nocite{张群2024在线,
张群2024出版,
Caplan1993-61-66,
Saito2006-169-176,
DESMARAIS1992-605-609,
Park2010-696-715,
Frese2013-378-398,
Myburg2014-356-362,SANTER2025ANN,SHINOTSUKA2023SAMPLE}



\textbf{示例:}

{\printbibliography[heading=none,env=indentegenv]}
\end{refsection}

\subsection{会议录}

\subsubsection{通用要求}

凡以图书、图书中的析出文献、连续出版物、连续出版物中的析出文献形式出现的会议录,其著录
项目与著录格式分别按8.2-8.5中的有关规则处理。除此之外的会议录根据本规则著录。

\subsubsection{著录项目}

主要责任者 有则必备;
题名 必备;
其他题名信息 有则必备;
文献类型标识 必备;
文献载体标识 电子资源必备;
会议地点 有则必备;
会议名称 有则必备;
会议年份 有则必备;
引文页码 有则必备;
获取和访问路径 电子资源必备;
永久标识符 电子资源可选。

\subsubsection{著录格式}
主要责任者. 题名:其他题名信息[文献类型标识/文献载体标识]. 会议地点:会议名称,会议年
份:引文页码. 获取和访问路径. 永久标识符.


\begin{refsection}
\nocite{王莉2023--,牛志明2012--,中国社会科学院台湾史研究中心2012--,肖希明2024民国}
\nocite{汪学军2002-22-25,贾东琴2011-45-52a,陈志勇2011--}
\nocite{WANG2022ACAIT,HU2024SSDBM,Yufin2000--}
\nocite{Babu2014--,FOURNEY1971-17-38}

{
 \hyphenpenalty=100 %断词阈值,值越大越不容易出现断词
 \tolerance=5000 %丑度,10000为最大无溢出盒子,参考the texbook 第6章

\textbf{示例:}

{\printbibliography[heading=none,env=indentegenv]}
}
\end{refsection}


\subsection{学位论文}

\subsubsection{著录项目}

主要责任者 有则必备;
题名 必备;
其他题名信息 有则必备;
文献类型标识 必备;
文献载体标识 电子资源必备;
学位授予单位所在地 有则必备;
学位授予单位 有则必备;
学位授予年 有则必备;
引文页码 有则必备;
获取和访问路径 电子资源必备;
永久标识符 电子资源可选。

\subsubsection{著录格式}

主要责任者. 题名:其他题名信息[文献类型标识/文献载体标识]. 学位授予单位所在地:学位授
予单位,学位授予年:引文页码. 获取和访问路径. 永久标识符.

\begin{refsection}

\nocite{王琦2022,CALMS1965--,何筱梅2016,CHRISTOU2024,曲恩熙2023社交}


\textbf{示例:}

{\printbibliography[heading=none,env=indentegenv]}
\end{refsection}


\subsection{报告}

\subsubsection{通用要求}

凡以图书、图书中的析出文献、连续出版物、连续出版物中的析出文献出现的报告,其著录项目与
著录格式分别按 8.2-8.5 中的有关规则处理。除此之外的报告根据本规则著录

\subsubsection{著录项目}

要责任者 有则必备;
题名 必备;
其他题名信息 有则必备;
报告编号 有则必备;
文献类型标识 必备;
文献载体标识 电子资源必备;
报告发布日期 有则必备;
引文页码 有则必备;
获取和访问路径 电子资源必备;
永久标识符 电子资源可选。

\subsubsection{著录格式}
主要责任者. 题名: 其他题名信息:报告编号[文献类型标识/文献载体标识]. 报告发布日期:引
文页码. 获取和访问路径. 永久标识符.

%\begin{tcolorbox}[colback=yellow!80,boxrule=0pt]
%注意:在线的时间,类图书出版物。
%\end{tcolorbox}

\begin{refsection}

\nocite{中国互联网络信息中心2012--}
\nocite{汤万金2013-09-30--,中国宏观2020蓝皮书,美国疾病2018黄皮书,中国信息2023白皮书}
\nocite{Calkin2011-8-9}
\nocite{DTFHA1990--}
\nocite{WHO1970--,UN2024DT}


\textbf{示例:}

{\printbibliography[heading=none,env=indentegenv]}
\end{refsection}


\subsection{标准}

\subsubsection{通用要求}

凡以图书、图书中的析出文献出现的标准,其著录项目与著录格式分别按 8.2-8.3 中的有关规则处
理。除此之外的标准根据本规则著录。

\subsubsection{著录项目}

主要责任者 有则必备;
题名 必备;
其他题名信息 有则必备;
标准号 有则必备;
文献类型标识 必备;
文献载体标识 电子资源必备;
其他责任者 可选;
版本 有则必备;
出版地 有则必备;
出版者 有则必备;
出版年 有则必备;
引文页码 有则必备;
获取和访问路径 电子资源必备;
永久标识符 电子资源可选。

\subsubsection{著录格式}
主要责任者. 题名:其他题名信息:标准号[文献类型标识/文献载体标识]. 其他责任者. 版本.
出版地:出版者,出版年:引文页码. 获取和访问路径. 永久标识符.

\begin{refsection}

\nocite{全国信息文献标准化技术委员会2010-3-3}
\nocite{全国广播电视标准化技术委员会2007-1-1}
\nocite{水电水利2020,国家环境保护局科技标准司1996-2-3,
中国铁建2021,华北水利2022,网络安全2021,国家标准局信息分类编码研究所1988-59-92}
\nocite{Auditdata2019,Softwareinterface2021,
Informationtechnology2020,Lithiumbattery2022,
Explosiveatmospheres2016,Atmospheresexplosives2016}


\textbf{示例:}

{\printbibliography[heading=none,env=indentegenv]}
\end{refsection}


\subsection{专利}

\subsubsection{通用要求}

凡以图书、图书中的析出文献、连续出版物、连续出版物中的析出文献出现的专利,其著录项目与
著录格式分别按 8.2-8.5 中的有关规则处理。除此之外的专利根据本规则著录。

\subsubsection{著录项目}

专利申请者/所有者 必备;
专利题名 必备;
其他题名信息 有则必备;
专利申请号 必备;
文献类型标识 必备;
文献载体标识 电子资源必备;
公告日期或公开日期 有则必备;
引文页码 有则必备;
获取和访问路径 电子资源必备;
永久标识符 电子资源可选。

\subsubsection{著录格式}

专利申请者/所有者. 专利题名:其他题名信息:专利申请号[文献类型标识/文献载体标识]. 公告
(公开)日期:引文页码. 获取和访问路径. 永久标识符.\footnote{非单个专利的汇编会按图书等类处理,但不容易区分,目前考虑利用publisher和url判断。当url不存在且publisher存在时,则输出publisher和到年份的日期。}


%\begin{tcolorbox}[colback=yellow!80,boxrule=0pt]
%注意:非单个专利的汇编类似图书如何考虑?。
%\end{tcolorbox}

\begin{refsection}

\nocite{邓一刚2006--,张凯军2012-04-05--,李华2023,
西安电子科技大学2002--,河北绿洲生态环境科技有限公司2001--,
苏州生物2022,Tachibana2005--,TRISCO2022,中国焊接协会2023}



{
 \hyphenpenalty=100 %断词阈值,值越大越不容易出现断词
 \tolerance=5000 %丑度,10000为最大无溢出盒子,参考the texbook 第6章

\textbf{示例:}

{\printbibliography[heading=none,env=indentegenv]}
}
\end{refsection}

\subsection{网站和网页}

\subsubsection{通用要求}

凡以图书、图书中的析出文献、连续出版物、连续出版物中的析出文献出现的专利,其著录项目与
著录格式分别按 8.2-8.5 中的有关规则处理。除此之外的专利根据本规则著录。

\subsubsection{著录项目}

网站的著录项目:
主要责任者 有则必备;
题名 必备;
文献类型标识 必备;
文献载体标识 必备;
创建或修改日期 有则必备;
引用日期 必备;
获取和访问路径 必备。

网页的著录项目:
主要责任者 有则必备;
题名 必备;
文献类型标识 必备;
文献载体标识 必备;
创建或修改日期 有则必备;
引用日期 必备;
获取和访问路径 必备;
永久标识符 可选。

\subsubsection{著录格式}

网站:主要责任者. 网站题名[文献类型标识/文献载体标识].(创建或修改日期) [引用日期]. 获取和访问路径.

网页:主要责任者.题名[文献类型标识/文献载体标识].(创建或修改日期) [引用日期]. 获取和访问路径. 永久标识符.


\begin{refsection}

\nocite{鲁迅博物馆2023,
BBC2020a,
BBC2020b}


{
 \hyphenpenalty=100 %断词阈值,值越大越不容易出现断词
 \tolerance=5000 %丑度,10000为最大无溢出盒子,参考the texbook 第6章


\textbf{网站示例:}

{\printbibliography[heading=none,env=indentegenv]}
}
\end{refsection}

\begin{refsection}

\nocite{北京市人民政府办公厅2005--,鲁迅博物馆2021,
杨立华2022,ISOHOME2020,ANTONIO2019,BEVINGTON2025,昨日之歌2015,
Zotero2024,仉尚航2024}


{
 \hyphenpenalty=100 %断词阈值,值越大越不容易出现断词
 \tolerance=5000 %丑度,10000为最大无溢出盒子,参考the texbook 第6章


\textbf{网页示例:}

{\printbibliography[heading=none,env=indentegenv]}
}
\end{refsection}


\subsection{档案}

\subsubsection{通用要求}

凡以图书、图书中的析出文献出现的档案,其著录项目与著录格式分别按 8.2-8.3 中的有关规则处
理。除此之外的档案根据本规则著录。

\subsubsection{著录项目}

主要责任者 有则必备;
题名 必备;
其他题名信息 有则必备;
档号 有则必备;
文献类型标识 必备;
文献载体标识 电子资源必备;
收藏地 有则必备;
收藏者 有则必备;
档案形成日期 有则必备;
引文页码 有则必备;
获取和访问路径 电子资源必备;
永久标识符 电子资源可选。

\subsubsection{著录格式}

主要责任者. 题名: 其他题名信息:档号[文献类型标识/文献载体标识]. 收藏地: 收藏者,档案形
成日期:引文页码. 获取和访问路径. 永久标识符.\footnote{考虑到档案形成日期与一般图书类文献的日期的差异,要对档案的类型做区分,从示例中看到有网址的通常可以不视作图书等一般类型。所以在判断时做如下处理:当存在网址时且没有出版地时,转成online处理,设置entrysubtype为archive,并给出档案形成日期;否则根据是否存在析出文献题目(booktitle)来判断是否是析出的文献,若是则转成inbook处理,否则按默认的archive处理(实际用manual类输出,其出版项与book类一致),并对档案形成日期做专门处理。}

%\begin{tcolorbox}[colback=yellow!80,boxrule=0pt]
%注意:通则中的格式如何实现,需要与正常情况区分。档案收藏日期如何处理?
%\end{tcolorbox}


\begin{refsection}
\nocite{李鸿章1887,湖北省建设厅1931,武汉市军事管制委员会1949,FITZWILLIAM1570,
中国第一历史档案馆2001--,清代奏折汇编2005,私立武昌2021}


\textbf{示例:}

{\printbibliography[heading=none,env=indentegenv]}

\end{refsection}


\subsection{地图}

\subsubsection{通用要求}

凡以图书、图书中的析出文献出现的地图册、地图集等,其著录项目与著录格式分别按8.2-8.3中
的有关规则处理。除此之外的地图根据本规则著录。

\subsubsection{著录项目}

主要责任者 有则必备;
题名 必备;
其他题名信息 有则必备;
比例尺 有则必备;
文献类型标识 必备;
文献载体标识 电子资源必备;
版本 有则必备;
出版地 有则必备;
出版者 有则必备;
出版年 有则必备;
尺寸 纸质单幅地图必备;
获取和访问路径 电子资源必备;
永久标识符 电子资源可选。

\subsubsection{著录格式}

主要责任者. 题名:其他题名信息. 比例尺[文献类型标识/文献载体标识]. 版本. 出版地:出版
者,出版年. 尺寸. 获取和访问路径. 永久标识符.\footnote{由于地图尺寸在biblatex中没有专门的域描述,所以用addendum域描述。版本则用version描述,而不是edition。}



\begin{refsection}
\nocite{胡健民2021--,刘祥沈2016地图,中工武大2019地图,吴自银2019地图,国家测绘2016地图,
訾冬梅2006地图,谭其骧1982地图,童世亨1926地图,CRIBB2015map}


\textbf{示例:}

{\printbibliography[heading=none,env=indentegenv]}

\end{refsection}


\subsection{数据集}

\subsubsection{通用要求}

凡以图书、图书中的析出文献、连续出版物、连续出版物中的析出文献形式出现的数据集,其著录
项目与著录格式分别按8.2-8.5中的有关规则处理。除此之外的数据集根据本规则著录。

\subsubsection{著录项目}

主要责任者 有则必备;
题名 必备;
文献类型标识 必备;
文献载体标识 电子资源必备;
版本 可选;
创建机构 电子资源可选;
创建或修改日期 有则必备;
引用日期 必备;
获取和访问路径 电子资源必备;
永久标识符 电子资源可选。

\subsubsection{著录格式}
主要责任者. 数据集名称[文献类型标识/文献载体标识]. 版本. 创建机构(创建或修改日期) [引
用日期]. 获取和访问路径. 永久标识符.\footnote{数据集若存在网址且来源不是期刊的转化为online处理,其他默认利用manual输出,根据来源按图书等类型输出,图书类与manual一致所以无需转换。但析出文献则需要转换,下面的示例中仅有第2条文献为期刊析出文献,所以目前仅实现了期刊析出文献的转换,考虑判断的方式是没有location且存在journal。}

\begin{refsection}

\nocite{草地数据1994--,李皓2024语料库,彭守璋2024降水,刘时银2012水川,仲晓雅2022夜间灯光,仲晓雅2022夜间灯光en,IHME2021Disease,郑涵2018生态}


\textbf{示例:}

{\printbibliography[heading=none,env=indentegenv]}

\end{refsection}


\subsection{预印本}


\subsubsection{著录项目}

主要责任者 有则必备;
题名 必备;
文献类型标识 必备;
文献载体标识 电子资源必备;
版本 有则必备;
预印本出版平台 可选;
创建或修改日期 有则必备;
引用日期 必备;
获取和访问路径 必备;
永久标识符 可选。

\subsubsection{著录格式}

主要责任者. 题名[文献类型标识/文献载体标识]. 版本. 预印本出版平台(创建或修改日期) [引
用日期]. 获取和访问路径. 永久标识符.\footnote{预印本通常是在线的,所以一般转为online处理,同时考虑一般的预印平台放在archiveprefix域中,将其转化为institution。并且根据该域信息设置misc类的标识符为PP。}

\begin{refsection}
\nocite{方向明2023元宇宙,肖玲2024数据,山东一医大2025,BLOSS2025trial,JENKINS2012light}


\textbf{示例:}

{\printbibliography[heading=none,env=indentegenv]}

\end{refsection}


\newpage
\appendix


\section{参考文献标引体系编制法}

\subsection{通用要求}

参考文献的引用体系可以按顺序编码制组织,也可以按著者-出版年制组织。引文参考文献既可以集中著录在文后或书末,也可以分散著录在页下端。阅读型参考文献著录在文后、书的各章节后或书末。

\subsection{顺序编码制}

\subsubsection{正文中的引用}

\paragraph{} 顺序编码制是按正文中引用的文献出现的先后顺序连续编码,将序号置于方括号中。如果顺序编码制用脚注方式时,序号可由计算机自动生成圈码



\begin{refsection}
\textbf{示例1:}引用单篇文献,序号置于方括号中

所谓移情,就是“说话人将自己认同于......他用句子所描述的时间或状态中的一个参与者”
\cite{Sunstein1996-903-903}。《汉语大词典》和张相
\cite{Morri2010--}都认为“可”是“痊愈”,
候精一认为是“减轻”\cite{罗杰斯2011-15-16}。......另外,根据候精一,表示病痛程度减轻的形容词“可”和表示逆转否定的副词“可”
是兼类词\cite{陈登原2000-29-29},这也说明二者应该存在着源流关系。

\textbf{示例2:}引用单篇文献,序号由计算机自动生成圈码

所谓移情,就是“说话人将自己认同于......他用句子所描述的时间或状态中的一个参与者”\footfullcite{Sunstein1996-903-903}。《汉语大词典》和张相
\footfullcite{Morri2010--}都认为“可”是“痊愈”,
候精一认为是“减轻”\footfullcite{罗杰斯2011-15-16}。......另外,根据候精一,表示病痛程度减轻的形容词“可”和表示逆转否定的副词“可”
是兼类词\footfullcite{陈登原2000-29-29},这也说明二者应该存在着源流关系。


\end{refsection}

\paragraph{} 同一处引用多篇文献时,应将各篇文献的序号在方括号内全部列出,各序号间用“,”。
如遇连续序号,起讫序号间用短横线连接。此规则不适用于用计算机自动编码的序号。

\begin{refsection}
\textbf{示例:}引用多篇文献

裴伟提出\cite{Humphrey1971--,KENNEDY1975-311-386}......

莫拉德对稳定区的研究
\cite{CRANE1972--,Weinstein1974-745-772,KENNEDY1975-311-386}......


\end{refsection}


\paragraph{} 多次引用同一著者的同一文献时,在正文中标注首次引用的文献序号,并在序号的“ []”外著录引文页码。如果用计算机自动编序号时,应重复著录参考文献,但参考文献表中的著录项目可简化为文献序号及引文页码,参见本条款的示例 2

\begin{refsection}
\textbf{示例1:}多次引用同一著者的同一文献的序号

……改变社会规范也可能存在类似的“二阶囚徒困境”问题:尽管改变旧的规范对所有人都好,
但个人理性选择使得没有人愿意率先违反旧的规范\cite{Sunstein1996-903-903}。
……事实上,古希腊对轴心时代思想真正的贡献不是来自对民主的赞扬,而是来自对民主制度的批评,苏格拉底、柏拉图和亚里士多德3位贤圣
都是民主制度的坚决反对者\pagescite[260]{Morri2010--}。
……柏拉图在西方世界的影响力是如此之大以至于有学者评论说,一切后世的思想都是一系列为柏拉图思想所作的脚注\cite{罗杰斯2011-15-16}。
……据《唐会要》记载,当时拆毁的寺院有4 600余所,招提、兰若等佛教建筑4万余所,没收寺产,并强迫僧尼还俗达260 500人。
佛教受到极大的打击\pagescite[326-329]{Morri2010--}。
……陈登原先生的考证是非常精确的,他印证了《春秋说题辞》“黍者绪也,故其立字,禾入米为黍,为酒以扶老,为酒以序尊卑,禾为柔物,亦宜养老”,指出:“以上谓等威之辨,尊卑之序,由于饮食荣辱。”\cite{陈登原2000-29-29}

\textbf{参考文献:}

\printbibliography[heading=none,env=indentegenv]
\end{refsection}

\begin{refsection}
\textbf{示例2:}多次引用同一著者的同一文献的脚注序号

……改变社会规范也可能存在类似的“二阶囚徒困境”问题:尽管改变旧的规范对所有人都好,但个人理性选择使得没有人愿意率先违反旧的规范
\footfullcite{Sunstein1996-903-903}。
……事实上,古希腊对轴心时代思想真正的贡献不是来自对民主的赞扬,而是来自对民主制度的批评,苏格拉底、柏拉图和亚里士多德3位贤圣
都是民主制度的坚决反对者\footfullcite[260]{Morri2010--}。
……柏拉图在西方世界的影响力是如此之大以至于有学者评论说,一切后世的思想都是一系列为柏拉图思想所作的脚注\footfullcite{罗杰斯2011-15-16}。
……据《唐会要》记载,当时拆毁的寺院有4 600余所,招提、兰若等佛教建筑4万余所,没收寺产,并强迫僧尼还俗达260 500人。
佛教受到极大的打击\footfullcite[326-329]{Morri2010--}。
……陈登原先生的考证是非常精确的,他印证了《春秋说题辞》“黍者绪也,故其立字,禾入米为黍,为酒以扶老,为酒以序尊卑,禾为柔物,亦宜养老”,指出:“以上谓等威之辨,尊卑之序,由于饮食荣辱。”\footfullcite{陈登原2000-29-29}
\end{refsection}

\subsubsection{参考文献表编制示例}

参考文献表采用顺序编码制组织时,各篇文献应按正文部分标注的序号依次列出(参见 A.2.1)。

\begin{refsection}
\textbf{示例:}

\nocite{Baker1995--,Chernik1982--,尼葛洛庞帝1996--,汪冰1997-16-16,杨宗英1996-24-29,Dowler1995-5-26}

\printbibliography[heading=none,env=indentegenv]
\end{refsection}

\end{document}
