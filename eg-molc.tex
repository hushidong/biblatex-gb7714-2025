
% !Mode:: "TeX:UTF-8"
% 用于测试gb7714-2025样式
\documentclass{article}
\usepackage[zihao=5]{ctex}
\usepackage[vmargin=3cm,left=1.5cm,right=4.5cm,marginparsep=7mm,
marginparwidth=3.3cm%,showframe
]{geometry}
\usepackage[backend=biber,style=chinese-molc]{biblatex}%sorting=nyt


%\usepackage{multicol}
\usepackage[colorlinks,bookmarksnumbered=true]{hyperref}
\usepackage{amsmath}
\usepackage{xcolor}
\usepackage{circledtext}
\circledtextset{resize=real,height=1.5ex,boxcolor=blue,charcolor=blue,boxlinewidth=1pt}
\usepackage{snotez}
\setsidenotes{
text-mark-format={\textsuperscript{\circledtext*[boxcolor=blue,charcolor=blue]{#1}}},
note-mark-format={\circledtext*[boxcolor=blue,charcolor=blue]{#1}}
}

\counterwithout{subsubsection}{subsection}
\usepackage[nostruts]{titlesec}
\titleformat{\section}{\zihao{4}\bfseries}{\chinese{section}}{1em}{}
\titleformat{\subsection}{\zihao{-4}\bfseries}{(\chinese{subsection})}{0pt}{}
\titleformat{\subsubsection}{\zihao{5}\bfseries}{\thesubsubsection}{1em}{}


%\BeforeBeginEnvironment{quote}{\begingroup}
%\AtBeginEnvironment{quote}{\fangsong\color{teal}}
%\AfterEndEnvironment{quote}{\endgroup}
\BeforeBeginEnvironment{quotation}{\begingroup}
\AtBeginEnvironment{quotation}{\fangsong\color{teal}}
\AfterEndEnvironment{quotation}{\endgroup}

\addbibresource{molc.bib}

\title{法学引注手册-文档复现}

\begin{document}

\maketitle


\section{示例}

\subsection{引用书籍的基本格式为:\sidenote{使用的entrytype为book,collection等。版本信息可以用edition保存,也可以用volume直接写。主编、编等角色可以用authortype或editortype填写。若是古籍则使用entrysubtype=classic。特殊的年份信息直接用year保存,除版本外的其他复杂信息也可以放到edition中。}}\label{sec:sub:typebook}

\begin{refsegment}
\nocite{molc.0:01,molc.0:02,molc.0:03,molc.0:04}%,萨利斯2000

\printbibliography[heading=none,segment=1]
\end{refsegment}


\subsection{引用已刊发文章的基本格式为:\sidenote{使用的entrytype为article和newspaper,后者对应报纸中的文章。从文集或书籍中析出的文献的entrytype可以使inbook或incollection等。article的volume 在年份后输出。inbook的volume和number在booktitle后输出。}}

\begin{refsegment}
%\nocite{molc.0:05,molc.0:06,molc2019.0:07,molc.0:08,molc.0:08a,molc.0:08b,molc.0:07}
\nocite{molc.0:05,molc.0:06,molc.0:07,molc.0:08}

\printbibliography[heading=none,segment=2]
\end{refsegment}


\subsection{引用网络文章的基本格式为:\sidenote{使用的entrytype为online。当引用的是网站时(用entrysubtype=website标记),网站名不加书名号,且前面需要加参见,访问日期后要加访问两字。微信公众号等信息也一同放在organization域中。}}

\begin{refsegment}
\nocite{molc.0:09,molc.0:10,molc.0:11,molc.0:12}

\printbibliography[heading=none,segment=3]
\end{refsegment}


\subsection{引用学位论文的基本格式为:\sidenote{学位论文使用的entrytype为thesis,也可以使用phdthesis或mastersthesis。}}


\begin{refsegment}
\nocite{molc.0:13}

\printbibliography[heading=none,segment=4]
\end{refsegment}


\subsection{引用法律文件的基本格式为:\sidenote{使用的entrytype为legislation,无逗号隔开的文件条款信息可以放到volume中,有逗号隔开的信息可以放到number里面。}}

\begin{refsegment}
\nocite{molc.0:14,molc.0:15}

\printbibliography[heading=none,segment=5]
\end{refsegment}


\subsection{引用司法案例的基本格式为:\sidenote{使用的entrytype为jurisdiction,案例的信息可以直接放在number中,若带有类似期刊的信息,也可按期刊文章的形式写。}}

\begin{refsegment}
\nocite{molc.0:16,molc.0:17}

\printbibliography[heading=none,segment=6]
\end{refsegment}



\subsection{引用英文报刊文章和书籍的基本格式为:\sidenote{著录格式见第
\ref{sec:fmt:engrefs}节。}}

\begin{refsegment}
\nocite{molc.0:18,molc.0:19,molc.0:20}

\printbibliography[heading=none,segment=7]
\end{refsegment}


\section{引注的一般规范}

\subsection{引注的基本要求}

\subsubsection{使用引注应当必要和适度}

\subsubsection{文献来源真实、 相关、权威}

\subsubsection{引注信息准确、完整、简洁}\label{sec:ssub:postnote}
已出版文献引注信息的内容, 原则上从原文原著。 书籍的作者、名称和出
版信息(如“修订版”“增订版”“第×版”),以版权页为准。 原文献的署名或
者标题有错误的,先依原文,之后可以在适当位置注明。
\sidenote{注明可通过为cite命令提供postnote信息实现,比如:
\textbackslash cite[prenote][postnote] \{entrykey\}。}

\begin{quotation}
\fullcite[][(引者注: 此处“马歇尔”系“马伯里”之误)]{朱苏力1998}
\end{quotation}


\subsection{引注的一般格式}

\subsubsection{引注信息的排版}
注释内容采用页下脚注。


引注符号使用阿拉伯数字,可以带圆圈或者六角括号,也可以不带。 对论
文作者的介绍、翻译作品的译者注,可以使用其他符号,以示区别。

\subsubsection{引注符号的位置}

对全句的引用,引注符号置于句号、问号等标点之后。 

\begin{quotation}
在《行政诉讼法》 起草过程中,关于受案范围问题曾有热烈的讨论。
\footnote{立法过程中的相关讨论和争鸣,参见\fullcite{金俊银1988,姜明安1988}}
尽管
多数学者主张概括规定法院应当受理的案件范围,以使受案范围尽量宽泛, 
\footnote{同上注。特别是俞梅荪、孙林文和张树义文,针对全国人大常委会公布的法律草案,明确主张采用概括式规定。}
立法最终采取了逐项列举的方式。比起此前各个单行法,《行政诉讼法》 规定的
受案范围“有所扩大” 
\footfullcite{王汉斌1989}
,但与概括规定的主张还相距很远。
\end{quotation}

\subsubsection{引文的编排处理}


大段引用, 或者有其他情况作者需要特别强调的, 引注内容可以独立成段,
变换字体,缩进编排。 例如:

\begin{quotation}
河南是全国行政诉讼大省, 10 年受案 10.6 万余件,约占全国的六分之一。
该省于 1996 年 4 月召开全省法院行政审判工作会议:
除各中院的院长、行政庭长和 1995 年行政诉讼、非诉行政案件收案‘双
超百’的 23 个基层法院的一把手和三个大力支持行政审判工作的县委
书记参加会议外(\textbf{以之鼓励先进!}),我们还让行政诉讼案件收案不足 10
件的 14 个基层法院的一把手参加会议(\textbf{以之鞭策后进!})。在省院会议
上,李道民院长(\textbf{针对各地收案悬殊的状况})要求每个法院的领导都要
认真查找一下原因,关键是从法院内部、从领导自身找原因;要认真反
思一下,自己思想是否解放,是不是真正重视行政审判工作。
\footfullcite{河南高院2000}
\end{quotation}


\subsubsection{文中图表的注释}

对文中图表的来源或者内容的注释,可以采用页下注, 也可以置于图表下
面; 置于图表下面的, 不与其他注释连续编码。

\subsubsection{文中夹注古籍}


引用常见古籍经典中的语句, 出处又相当简短的,可以在正文中使用夹注,
以代替页下脚注。 夹注一般只标书名和篇名,用中圆点连接,用圆括号括注,
紧随引文之后。 例如:

\begin{quotation}
天神所具有的道德意志,代表的是人民的意志。这也就是所谓“天聪明自
我民聪明,天明畏自我民明畏”\citejz{尚书皋陶谟},“民之所欲,天必从之”
\citejz{尚书泰誓}。
\end{quotation}


\subsubsection{文中夹注外文}

正文中提及的外国人名、地名和重要术语, 读者不熟悉或者容易误解的,
第一次出现时, 在正文中夹注外文。 例如:

\begin{quotation}
自毕克尔(A. Bickel)提出司法审查“反多数难题”(counter-majoritarian
difficulty)\footfullcite{Bickel1962},该问题占据了美国宪法研究的中心,无数的笔墨花在对司法审查
合法性的探讨上。
\end{quotation}

\subsubsection{文中夹注页码}

一般来说,不鼓励在一个篇章中频繁、密集引用同一文献。对特定书籍和
文章的专门介绍、评论、商榷,确需多次引用的,可以在适当声明后,在正文
相应位置用括号夹注页码。

\subsubsection{引领词的用法}

一般来说,概括引用可以使用“参见”引领,直接引用原文用“见”。

同一文献有不同出处,需要互相印证的,可以写“又见”。


\subsubsection{标点符号的用法}

标点符号的使用应当遵循国家标准《标点符号用法》(GB/T 15834-2011),
防止误用。

\subsubsection{同一注释包含多个文献}
同一注释里包含多条同类文献的,一般按时间顺序排列,用分号隔开。例
如:

\begin{quotation}
\fullcite{马怀德2004,沈岿2000,molc.0:08}
%马怀德主编:《司法改革与行政诉讼制度的完善》,中国政法大学出版社 2004
%年版; 胡建淼主编:《行政诉讼法修改研究》,浙江大学出版社 2007 年版; 杨小
%君主编:《行政诉讼法问题研究与制度改革》,中国人民公安大学出版社 2007 年
%版; 莫于川主编:《建设法治政府需要司法更给力》,清华大学出版社 2014 年
%版; 何海波等:《理想的行政诉讼法》, 载《行政法学研究》 2014 年第 2 期。
\end{quotation}

同一注释里中外文文献混合排列的,结尾句号使用最后文献的语种。例如:
\begin{quotation}
参见\fullcite{沈岿2000,Kellogg2007}
%沈岿:《制度变迁与法官的规则选择:立足刘燕文案的初步探索》, 载
%《 北大法律评论》 第 3 卷第 2 辑, 法律出版社 2000 年版; Thomas Kellogg,
%“Courageous Explorers”? Education Litigation and Judicial Innovation in China, 20
%Harvard Human Rights Journal 141 (2007).
\end{quotation}

外文文献包含在一个句子中,整个句子属于中文句式的,结尾句号使用中
文句号。例如:
\begin{quotation}
对于“分离的领域”的经典论述,参见\fullinnercite{Bradwell1892}。
%Bradwell v. Illionois, 83 U.S. 130, 141 (1892)。
\end{quotation}


\subsubsection{同一文献多次出现}

同一文献在文中多次出现的,第一次必须引用完整信息, 再次引用时可以
略写。 是否略写,由各刊物、出版社或者出版社授权编辑自行决定。为防止修
改、编辑过程中发生错乱,作者投稿时不建议采用略写。
\sidenote{为了可以让用户自由控制是否进行略写,区分不同的引用命令。使用footfullcite命令除非前一条文献相同否则不略写,而footbriefcite命令则主动缩写。}

\footfullcite{molc.14:16}
\footnote{插入脚注}
\footbriefcite{molc.14:16}
\footbriefcite{molc.14:16}
\footnote{同前注, 应松年、马怀德书,第 330 页。}

前后紧邻的两个引注,文献完全相同,而且没有其他文献干扰的,可以写
“同上注” 或者“Ibid”;所引文献是外文的,从该语种习惯。 
\sidenote{Ibid. refers to the same author and source (e.g., book, journal) in the immediately preceding reference.
op. cit. refers to the reference listed earlier by the same author.}
例如:

\footfullcite{RvPanel1987}
\footbriefcite{RvPanel1987}


\subsubsection{同一文献多个来源}

同一文献有多个来源的,原则上只引用一个来源,即最早的出处。一些早
期文献的最早出处一般读者不易查找,作者认为有必要的,可以同时引注该文
献重印或者转载的信息。\sidenote{额外的来源可以通过附加信息来实现,将附加的信息填入addendum域即可。}
例如:

\begin{quotation}
\fullcite{顾颉刚1930}


\fullcite{熊元翰1914}
\end{quotation}


\subsubsection{对文献的解释和评论}
除了纯粹的解释性脚注,引用文献的脚注也可以适当夹带解释或者评论。

1) 对文献内容的解释。例如:

\begin{quotation}
\fullcite{RvPanel1987} 该案涉及对一个证券交易机构的司法审查。这个交易机构既非行政机关也没有法律授权,却行使规制和惩罚的职能。
\end{quotation}

2) 提示类似研究或者相反观点。例如:

\begin{quotation}
\fullcite{Cohen1986} 类似的研究还有 \fullcite{Quirk1981}
\end{quotation}

3) 对文献整体的评论。 例如:

\begin{quotation}
这些意见没有公开,但从最高人民法院法官的著作中,可以了解法院的基
本立场。参见\fullcite{江必新2005,江必新2013,李广宇2013}
\end{quotation}

\subsection{与引注有关的论文部件}

\subsubsection{论文标题}

\subsubsection{论文摘要}

\subsubsection{关键词}

\subsubsection{作者介绍}

\subsubsection{项目说明}

\subsubsection{作者致谢}

\subsubsection{参考文献}

\section{中文引注体例}

\subsection{引用纸质出版文献}

\subsubsection{引用纸质出版文献的基本要求}

\subsubsection{原创作品的作者}

作者对于作品的形成注入了较多贡献的,在引用时视为原创作品;主要是
翻译、整理、校勘他人文字和口头作品的,在引用时视为原创作品。

\begin{quotation}
\fullcite{陈卫佐2015}
\end{quotation}

\subsubsection{编辑作品的编者}

编辑作品的, 在主要作者(编者) 姓名之后, 加“主编”“编”“编著”“编
译”“译注”等,说明文献性质。例如:

\begin{quotation}
\fullcite{何海波2018}
\end{quotation}

\subsubsection{翻译作品的译者}

翻译作品的译者位于文献名称之后、出版信息之前。
%[美]富勒:《法律的道德性》,郑戈译,商务印书馆 2005 年版。
%[美]欧中坦:《 千方百计上京城:清朝的京控》,谢鹏程译, 载高道蕴等编:
%《美国学者论中国法律传统》,中国政法大学出版社 1994 年版。
翻译作品有校对者的,可以视情况写明校对者。
\sidenote{译者、校对者等信息较多时可以用团体作者的方式直接放到一个域比如editor中,并利用editortype给出最后的责任者的角色。}
例如:
%[英]科林· 斯科特:《 规制、治理与法律》, 安永康译, 宋华琳校, 清华大
%学出版社 2018 年版。

\begin{quotation}
\fullcite{molc.0:04}

\fullcite{molc2019.0:07}

\fullcite{斯科特2018}

\fullcite{斯科特2018a}
\end{quotation}


\subsubsection{口述作品的整理者}
口述作品,口述者位于文献名称之前, 整理者位于文献名称之后、出版信
息之前。例如:

\begin{quotation}
\fullcite{江平2010}
\end{quotation}

\subsubsection{勘校作品的校对者}
古籍点校作品, 可以视情况写明点校者;早期作品再版的,也可以视情况
写明校对者。例如:

\begin{quotation}
\fullcite{沈家本1985}
\end{quotation}

\subsubsection{作者的一般写法}

引用著作直写作者姓名,不写职务。 
作者姓名、名称,原则上应当按照版权声明标注全名。
作者系多人,以“编写组”“编委会”等为名的,从其署名。
作者使用笔名的,从笔名;必要时,可以括注真名。例如:

\begin{quotation}
\fullcite{慕槐1995}
\end{quotation}


\subsubsection{外籍作者}
外籍作者,在姓名之前用方括号[ ]注明国籍;台港澳作者, 无需特别标明。

\subsubsection{合作作品的作者}
%\textcolor{red}{pagetotal,url要处理。}
合作作品的几位作者之间,用顿号间隔。

\begin{quotation}
\fullcite{molc.32:02}
\end{quotation}

\subsubsection{作者信息的省略}
常用基本典籍和官修大型典籍, 可不标注作者。例如,《论语》《资治通鉴》、
二十四史等。由编委会组织编写的辞典、百科全书等,可以省略作者。被引文献没有作者署名,经过查考仍无法确定作者的,不写作者姓名,或
者只写“佚名”。书名包含作者姓名的个人文集, 从书名可以直接推断作者的, 可以省略文
集作者。例如:

\begin{quotation}
\fullcite{汉语词典}

\fullcite{邓小平1994}
\end{quotation}


\subsubsection{文献名称}
1)引用文献的名称,包括文章标题, 用书名号; 栏目名称、丛书名,可视
情况使用引号
2)引用文献应当用全称,不用简称。文献名称冗长的,第一次引用仍应全
称,再次引用时可以略称( 详见本手册第 12 条)。
3)所引文献标题带有逗号、问号等标点符号的,从原文,不省略。例如:
《执行难,难于上青天?》
4)所引文献有副标题的, 主标题和副标题之间的符号( 冒号、 破折号等),
一般从原文。


\subsubsection{报纸的标题}
所引文章标题包含两个部分,但没有主从关系的,用一个字符的空格隔开。

\subsubsection{图书的标题}
所引图书包含两个不同主题, 互相没有主从关系的,用一个字符的空格隔
开。


\subsubsection{图书的版本}
1)图书版本不同于印刷次数;同一版本多次印刷的,仍为一版。版本信息
以版权页为准。
2) 图书初版的, 无需标明“初版”“第 1 版”; 再版的, 在图书名称后, 括
注“修订版”“增订版”“第×版” 等。
%张新宝:《 侵权责任法》, 中国人民大学出版社 2006 年版。
%张新宝:《 侵权责任法》( 第 4 版), 中国人民大学出版社 2016 年版。
3)外文图书的中译本一般无需标明原书版次;确有必要时, 可以在书名后
用括号标明“原书第× 版”。
%[美]理查德· J.皮尔斯:《 行政法》( 原书第 5 版), 苏苗罕译,中国人民大
%学出版社 2016 年版。
%[英]劳特派特修订:《奥本海国际法》(第 8 版上卷第一分册),王铁崖、陈
%体强译,商务印书馆 1971 年版。
4)图书再版时变更出版社,没有标明再版的,从原书信息。 例如:


\begin{quotation}
\fullcite{张新宝2006}

\fullcite{molc.0:02}
\end{quotation}

\subsubsection{图书的出版信息}

1) 图书的出版信息,包括出版社和出版时间。
2) 出版社名称应当完整,出版社之前不写所在城市。例如,只写“法律出
版社”,不写“ 北京:法律出版社”。
3) 两家出版机构联合出版的,应当一一列明,中间用顿号。 
%魏振瀛主编:《 民法》(第 7 版), 北京大学出版社、高等教育出版社 2017
%年版。
4) 出版时间只写年、不写月。 例如:

\begin{quotation}
\fullcite{魏振瀛2017}
\end{quotation}


\subsubsection{期刊的出版信息}

1)期刊名称用书名号。期刊的“社会科学版”“人文社会科学版”属于期
刊名称的一部分,用括号置于书名号内。 
2)期刊名称有变化的, 写所引文献发表当时的期刊名。
3)期刊有别名的, 写主要名称。

\subsubsection{其他连续出版物的出版信息}
期刊之外的其他连续出版物(包括一卷多辑、连续页码的出版物), 一般标
明主编或者编辑, 直接标注“第×卷”“第× 辑” 或者“第× 卷第× 辑”,后面
注明出版社和出版年份。 连续出版物的封面未标明主编的,引注时也不标主编。
例如:

\begin{quotation}
\fullcite{molc.0:06}

\fullcite{沈岿2000}
\end{quotation}


\subsubsection{文集的出版信息}
引用会议文集、纪念文集或者其他专题文集中的文章,应当完整标注该书
籍的编者、书名和出版信息,以“载”字开头。
\begin{quotation}
\fullcite{宋炉安2006}
\end{quotation}


\subsubsection{报纸和新闻类杂志的出版信息}
报纸的出版信息,一般注明年、月、日;必要时,注明版次。
%例如:
%何海波:《判决书上网》, 载《法制日报》 2000 年 5 月 21 日,第 2 版。
新闻类杂志的出版信息,一般注明期次,必要时括注刊发时间。
\sidenote{刊发时间可以如第\ref{sec:ssub:postnote}节一般利用postnote提供。比如:
\textbackslash fullcite[(2009 年 6 月 22 日)]\{王和岩1988\}}
例如:

\begin{quotation}
\fullcite{molc.0:08}

\fullcite[(2009 年 6 月 22 日)]{王和岩1988}
\end{quotation}


\subsubsection{古籍的出版信息}
传统的刻本、抄本, 应当标明版本信息; 现代出版的标点本、整理本、影印
本, 也可根据需要标注出版方式。
%姚际恒:《古今伪书考》卷三,光绪三年( 1877 年) 苏州文学山房活字本
%《太平御览》卷六九○,中华书局 1985 年影印本,第 3 册,第 3080 页下
%栏。
引用常用基本典籍,不涉及内容争议的,可以省略出版信息。
\sidenote{关于古籍的出版信息的填写说明见第\ref{sec:sub:typebook}节。}
例如:
%《论语· 述而篇》


\begin{quotation}
\fullcite{姚际恒1877}


\fullcite{太平御览1985}


\fullcite{论语述而}

\end{quotation}


\subsubsection{台湾文献的出版信息}
引用我国台湾地区的文献应当遵守“一个中国”原则,一般不使用“国立”
“中央”等字眼。\sidenote{台湾文献除了去除国立之外,其他的实际上没必要区分,统一就行。}

\subsubsection{页码}

1) 引用书籍或者论文特定部分的内容,应当标明页码;如果是概括提及书
籍、 论文整体,不标页码。 
%例如:
%瞿同祖:《中国法律与中国社会》,商务印书馆 2010 年版, 第 5-30 页。
%崔国斌:《知识产权法官造法批判》, 载《中国法学》 2006 年第 1 期,第 163 页。
2) 古籍刻本的页码有两面的, 可以进一步用 a、 b 标明。例如:
%姚际恒:《古今伪书考》卷三,光绪三年( 1877 年) 苏州文学山房活字本,
%第 9 页 a。
3) 引用同一著作的几个内容互不连续的页码,用“、” 隔开;内容连续的
页码(包括连续两页), 用短横线“ -” 连接。 页码数字置于“第”和“页”中
间,“第”和“页”只写一次。\sidenote{页码信息可以利用pages域提供,也可以利用postnote在引用时提供。}
例如:

\begin{quotation}
\fullcite{molc.0:02}


\fullcite[,第 9 页 a]{姚际恒1877}


\fullcite[,第 55、 64-68 页]{molc.0:04}

\end{quotation}


\subsubsection{章节}
标示页码,一般写起始页和结束页。 必要时,可以在页码后括注提示相应
内容。
%例如:
%《中国大百科全书·法学》,大百科全书出版社 1984 年, 第 81 页(“法的
%解释”条)
在一些场合下, 引用书籍特定章节更加明白的,可以只标明章节序号和名
称,不写页码。
\sidenote{章节信息可以放在chapter域中,也可以利用引用时的postnote信息提供。}
例如:
%《圣经· 出埃及记》, 20:3。
%《元典章》卷一九《 户部五· 田宅· 家财》, “ 过房子与庶子 分家财” 条。
%应松年主编:《当代中国行政法》,人民出版社 2018 年版, 第二章“行政法
%的渊源”。

\begin{quotation}
\fullcite{molc.0:03}
\end{quotation}

\subsection{引用网络、 电视和音像制品}

\subsubsection{引用网络电视文献的原则}
引用网络电视文献应当谨慎。

\subsubsection{引用网络文献}
引用互联网上的文献,引领词、作者、文章名参照前述做法。
\begin{quotation}

\fullcite{molc.0:09}

\fullcite{molc.0:10}
\end{quotation}


\subsubsection{引用个人博客、 微信公众号}
引用个人博客、微信公众号等自媒体,应当非常谨慎。


\subsubsection{引用电视节目}

引用电视节目,应当标明电视台和电视栏目名称、播出时间;必要时,可
以标明节目主持人姓名;可能的话,标明该电视节目在互联网上的链接。
\sidenote{电视节目的entrytype可以用video,播出信息可以放到version域中,网址放到url域中。}
比如:
\begin{quotation}
\fullcite{袭警案2008}
\end{quotation}

\subsubsection{引用音像制品}
引用 CD、 DVD 等介质的音像制品,应当标明其名称、制作单位和发行时间。


\subsection{引用未发表文献\sidenote{未发表文献除了会议、学位论文等常规文献,使用unpublished作为其entrytype。}}

\subsubsection{引用访谈}
访谈应当事先获得对方同意。 引用访谈,应当标明访谈的时间、地点或者
方式。例如:

\begin{quotation}
\fullcite{访谈2000}
\end{quotation}

\subsubsection{引用私人通讯}
引用私人通讯原则上应获得对方同意,并标明通讯方式和通讯时间。例如:

\begin{quotation}
\fullcite{邮件2002}
\end{quotation}

\subsubsection{引用内部资料}
引用未发表工作报告、调研报告或者口头讲话, 原则上应征得当事人同意
(不涉及秘密的官方报告和会议讨论除外),并适当注明文献产生、保管或者
公开之时间、地点和方式。 
\sidenote{默认情况下unpublished没有author只有title,所以不加书名号,当存在作者时,则加上书名号。}
例如:
\begin{quotation}
\fullcite{行政庭2001}

\fullcite{座谈会2009}
\end{quotation}



\subsubsection{引用会议论文}
引用未发表的会议论文,一般应当经作者同意。如果会议论文明确要求“请
勿援引”,则不应援引,除非得到作者特别许可。
引用会议论文,应当注明会议名称、 时间、 地点等会议信息;持续举行的
年会, 时间、地点可以从略。
\sidenote{所有的会议信息可以一起放到publisher中,而不是放到booktitle中。}
例如:
\begin{quotation}
\fullcite{贺卫方2001}
\end{quotation}


\subsubsection{引用学位论文}
引用已公开的学位论文, 无需经过作者同意。该学位论文已经发表或者修
改后发表的,一般应当引用发表后的文献。
\begin{quotation}
\fullcite{molc.0:13}
\end{quotation}


\subsubsection{引用档案文献}
引用档案文献,需标明文献名称、形成时间、 保管机构、 档案编号。
\sidenote{档案的entrytype使用archive,档案信息放在publisher内,档案号放在number内。}
例如:
%雷经天:《 关于边区司法工作检查情形》( 1943 年 9 月 3 日), 陕西省档案
%馆藏陕甘宁边区高等法院档案,档案号 15/149。

\begin{quotation}
\fullcite{雷经天1943}
\end{quotation}


\subsection{引用法律文件}

\subsubsection{法律文件的名称}
%\textcolor{red}{url要处理}
1) 法律文件名称应加书名号。
2) 法律文件的“试行”“草案”, 以及刑法修正案的序号, 应当视为法律
文件名称的一部分,括注于书名号内。

\begin{quotation}
\fullcite{molc.58.2:02}
\end{quotation}



\subsubsection{法律文件名称的缩写}
在不引起误解的情况下, 法律文件名称中的“中华人民共和国” 可以省略,
无需特别说明。 


\subsubsection{法律文本的版本}
1) 引用经过修改的法律文件,应当注明所引版本的制定、修改年份, 除
非正文已经交代或者根据情境不难判断。
%例如:
%《公司法》( 2005 年修订)第 16 条。
%《公司法》( 2013 年修正)第 36 条。
2) 引用已经失效的法律文件,应当予以注明,除非正文已经交代或者根
据情境不难判断。
例如:
%《 最高人民法院、最高人民检察院关于依法严惩破坏计划生育犯罪活动的
%通知》(已废止), 法发〔 1993〕 36 号。

\begin{quotation}
\fullcite{molc.60.1:01}

\fullcite{molc.60.2:01}
\end{quotation}

\subsubsection{法律文件的条款序数}

1)为使行文简洁, 法律文本的条、款、项序数采用阿拉伯数字, 序数中
的括号省略。
%例如:
%《民法总则》第 27 条第 2 款第 3 项。
%《行政诉讼法》( 1989 年) 第 54 条第 2 项第 3 目。
%根据《 刑法》 第 64 条和《最高人民法院关于适用〈中华人民共和国刑事
%诉讼法〉的解释》第 138 条、第 139 条的规定,被告人非法占有、处置被害
%人财产的,应当依法予以追缴或者责令退赔。
2)原文引用法律文本的, 条、款、项、目的序数一般从原文。 即, 条、
款序数一般用汉字,项的序数用汉字加括号,目的序数用阿拉伯数字。
%例
%如:
%《民法总则》第二十七条第二款第(三)项。
%《行政诉讼法》( 1989 年) 第五十四条第(二)项第 3 目。
3)在任何情况下, 法律文件名称中的条款序数不得改为阿拉伯数字。 例
%如:
%《 最高人民法院关于适用刑法第六十四条有关问题的批复》
%《全国人民代表大会常务委员会关于〈 中华人民共和国民法通则〉 第九十
%九条第一款、〈 中华人民共和国婚姻法〉 第二十二条的解释》


\subsubsection{法律条文的排版}

法律条文包含多个款项,需要完整引用原文并且需要突出内容的,可以将
相关款项单独排列、分行分段。

\subsubsection{引用法律、法规、规章}
1) 援引最高立法机关的法律条文,一般只需提及该法的名称和条文序数。
2) 引用全国人大及其常委会通过的法律性质的决定,应当标明决定机关、
决定名称、决定时间和会议届次。
%例如:
%《 全国人民代表大会常务委员会关于严禁卖淫嫖娼的决定》, 1991 年 9 月
%4 日第七届全国人大常委会第二十一次会议通过。
3)援引法规、规章的条文,参照法律。必要时,可以进一步标明该法
规、规章的制定机关和年份。


\subsubsection{引用规范性文件}

1) 引用规范性文件,应当标明该文件的制定机关和文件号;必要时,进
一步标明发布日期。 文件号中的年份加六角括号〔 〕 。 
2) 规范性文件的名称包括制定机关的,制定机关在书名号内;否则,制
定机关放在书名号前。
3) 文件号一般在文件名之后,用逗号分隔。 文章叙述中提及规范性文件
又需要标明文件号的, 为保持行文顺畅, 可以在文件名之后括注文件号。
4) 一些较早时期发布的规范性文件,读者难以查找的,最好进一步标明
可供查阅的载体。\sidenote{一些查阅信息可以一股脑放到number域里面,也可以直接在正文表明。同时为避免出现文献末尾的标点,则使用fullinnercite命令引用文献。}
例如:

\begin{quotation}
\fullcite{molc.0:15}

\fullinnercite{molc.0:15a},2007 年 7 月 11 日发布。

国务院下发的\fullcite{molc.0:15}

\citetitle{molc.0:15}(国发〔2007〕19号)明确要求,“2007年在全国建立农村最低生活保障制度” 。

\fullcite{缺点批复1998}

\fullinnercite{缺点批复1998a},载司法部编《中华人民共和国司法行政规章汇编( 1949— 1985)》,法律出版社 1998 年版,第 646 页。
\end{quotation}


\subsubsection{引用国家标准}
引用国家标准,写明发布机关、名称和标准号;必要时,注明发布时间。
例如:

\begin{quotation}
\fullcite{molc.65:01}
\end{quotation}

\subsubsection{引用立法说明}
引用官方的立法说明,应当标明报告人、报告名称和报告场合;必要时,
可以用括号标明报告人的身份。

\subsubsection{引用会议决议}
引用官方会议的决议,写明决议名称、决议机关和作出决议的时间。例
如:
\begin{quotation}
\fullcite{王汉斌1989}
\end{quotation}


\subsubsection{引用外国法律和国际公约}
1) 引用外国法律或者国际公约的中文版本,视情况加国别(国际组织)
和年份。
%例如:
%英国《 1996 年仲裁法》
%美国《 统一买卖法》( 1906 年), 或者美国 1906 年《 统一买卖法》
2) 外国法律或者国际公约加书名号, 约定俗成的简称除外。
%例如, 《美
%利坚合众国宪法》 第十四修正案, 通常无需说明, 直接表述为:
%美国宪法第十四修正案
3) 国别或者国际组织一般置于书名号之前, 但国名或者国际组织名称是
法律文件名称一部分的除外。 
%例如:
%美国《 统一买卖法》,《法国民法典》
%联合国《儿童权利公约》,《联合国海洋法公约》
4) 引用外国法律或者国际公约的中文版本, 必要时可以括注外文。 
%例
%如:
%《联合国海洋法公约》( United Nations Convention on the Law of the Sea)38
5) 引用外国法律和国际公约的特定译本, 一般应当注明译者和出版信
息;引用国际公约的官方文本,不需要注明译者和出版信息。
%例如:
%陈卫佐译注:《德国民法典》(第 4 版),法律出版社 2015 年版, 第 1408 条
%“夫妻财产合同、契约自由”。
%《美国法典· 宪法行政法卷》, 中国社会科学出版社 1993 年版,第 276 页
%(第 554 条“裁决”)。
%联合国《儿童权利公约》第 27 条
6)引用外国法律和国际公约,条文序号原则上用阿拉伯数字,款项依习
惯用数字或者字母。 外国法律增订条文“之一”“ 之二”,不改为阿拉伯数
字。\sidenote{外国法律和国际公约的附加信息可以视情放在volume,pubstate,number,chapter,pages等域中。}

\subsubsection{引用台湾地区的法律文件}
引用我国台湾地区的法律文件,应当根据情境注明“我国台湾地区”或者
“台湾地区”。 引用台湾地区的“宪法”以及其他涉及两岸关系、容易产生“两
个中国”或者“一中一台”嫌疑的法律文件,必须打上引号; 其他法律文件,
使用引号或者做其他适当处理。 
%例如:
%台湾地区“民法” 第 12 条规定:“满二十岁为成年。”
%驾驶人无过失及情节轻微之肇事逃逸案,“司法院大法官会议” 释字第 777
%号, 2019 年 5 月 31 日公布。


\subsection{引用司法案例}

\subsubsection{案例名称}

民事和行政案件的名称,格式为:“×××(原告)诉×××(被告)×
××(案由)案”。 刑事案件的名称,格式为:“×××(被告人)×××
(指控罪名)案”。 必要时,可以加上审级说明,例如“×××上诉(再审)
案件”。

\subsubsection{案件文号}
1) 裁判文书一般注明审判法院、文书名称和案号。审判法院用全名,案
号中的年份加圆括号“( )”。 
%例如:
%北京市海淀区人民法院行政判决书,( 1998)海行初字第 142 号。
2)案号一般置于文书名称之后,用逗号隔开。 在有其他案件信息的情况下,
为使表述更加紧凑,可以把案号置于审判法院和文书名称中间,不用逗号。

\begin{quotation}
\fullcite{molc.0:16}
\end{quotation}


\subsubsection{案例来源}
最高法院、最高检察院发布的指导性案例,一般只标注发布机关、 指导
性案例的序号,并用括号标注发布年份。
%例如:
%荣宝英诉王阳、永诚财产保险股份有限公司江阴支公司机动车交通事故责
%任纠纷案, 最高人民法院指导案例 24 号( 2014 年)。
《最高人民法院公报》上的案例,可以只援引《公报》,不再引用裁判
文书(除非需要文本比较)。 
%例如:
%陆红霞诉南通市发展和改革委员会政府信息公开答复案,《最高人民法院公
%报》 2015 年第 11 期。
\begin{quotation}
\fullcite{molc.72:01a}

\fullcite{molc.0:17}
\end{quotation}


\subsubsection{裁判时间}
裁判时间一般无需标明。如果有必要的话,例如裁判年份与立案年份不
一致或者讨论涉及裁判时点,可以标明裁判时间。

\begin{quotation}
\fullcite{molc.73:01}

\end{quotation}

\subsection{引用统计数据}


\subsubsection{应当标明出处的数据}

除业内周知的事实外, 统计数据应当标明出处。

\begin{quotation}
与每年400万到600万件涉及行政争议的信访案件相比,行政诉讼案件简直
微不足道。\footfullcite{孙乾2014}与国外相比,中国的行政诉讼案件也是少得出奇:法国六千万
人口,地方行政法院一年受理的案件也近20万;德国八千万人口,几套法院
一年受理的各类行政性案件更是高达50万左右。
\end{quotation}

\subsubsection{引用统计数据的注意事项}
统计数据的来源应当可靠,数据应当合理。对于不可靠的来源、不合理的
数字,应当作出解释和评估。

\subsubsection{统计数据的图表呈现}
为更加直观地显示统计数据,可以使用图表。
图表应当与统计数据相符,力求直观和美观。


\section{外文引注体例}

\subsection{英文引注体例}\label{sec:fmt:engrefs}


\subsubsection{引用外文文献的一般要求}
中文文献优先引用。


\subsubsection{期刊文章基本格式为:\sidenote{期刊名、卷、页码等有两种方式,一种是卷和页码放在期刊名两侧,另一种是期刊名称之后写期刊卷数、首页页码。由选项gbenArtVolahead控制,默认为true,即使用第一种方式。}}

%比如:Stephen J. Choi \& Adam C. \textit{Pritchard, Behavioral Economics and the SEC}, Stanford Law Review, Vol.56:1, p.1-73 (2003)

\begin{refsegment}
\nocite{molc.0:18,molc.0:19,molc.78:02,molc.78:03}

\printbibliography[heading=none,segment=8]
\end{refsegment}


\subsubsection{报纸文章基本格式为:\sidenote{采用方式为:期刊名称之后出版年月日完整,版次若非数字需加at。}}

%比如:Andrew Rosenthal, \textit{White House Tutors Kremlin in How a Presidency Works}, New York Times, June 15, 1990, at A1

\begin{refsegment}
\nocite{molc.79:01}

\printbibliography[heading=none,segment=9]
\end{refsegment}

\subsubsection{书籍基本格式为:\sidenote{编辑作品,在编者之后加 ed.;两人以上编辑的,在编者之后加 eds.。翻译作品,在书名后加译者信息,译者前加 translated by。}}

\begin{refsegment}
\nocite{molc.0:20,molc.80:02}

\printbibliography[heading=none,segment=10]
\end{refsegment}


\subsubsection{集合作品中的文章基本格式为:\sidenote{用 in 表示“载于”。}}

\begin{refsegment}
\nocite{molc.81:01}

\printbibliography[heading=none,segment=11]
\end{refsegment}


\subsubsection{法规基本格式为:\sidenote{有逗号隔开的条款、来源等信息可以都放到number里面。}}

\begin{refsegment}
\nocite{molc.82:01,molc.82:02}

\printbibliography[heading=none,segment=12]
\end{refsegment}


\subsubsection{案例基本格式为:\sidenote{卷,页码,年份正常放在volume,pages,date中。当一些信息不便于单独保存时,可直接放到title内,若与标题间有逗号,则可以放到number里面。}}

\begin{refsegment}
\nocite{molc.83:01-1,molc.83:01-2,molc.83:02,molc.83:03}

\printbibliography[heading=none,segment=13]
\end{refsegment}


\subsubsection{英国法院案例}
英国早期的案例报告比较杂, 都为私人编撰, 引用时根据具体情况。



\subsubsection{网络文章基本格式为:\sidenote{日期为有月份名的日期。}}

\begin{refsegment}
\nocite{molc.85:01}

\printbibliography[heading=none,segment=14]
\end{refsegment}


\subsection{法文引注体例}

\subsubsection{法文学术文献}

\begin{quotation}
\fullcite{molc.86.1:01}

\fullcite{molc.86.2:01}

\fullcite{molc.86.2:02}

\fullcite{molc.86.3:01}

\fullcite{molc.86.4:01}

\fullcite{molc.86.5:01}

\fullcite{molc.86.6:01}
\end{quotation}


\subsubsection{法国法律}


\subsubsection{法国法院判决}


\subsubsection{法文网络信息}

\begin{quotation}
\fullcite{molc.89:01}

\end{quotation}



\subsubsection{引注关系}

\subsection{德文引注体例}

\subsubsection{德文学术文献}

\begin{quotation}
\fullcite{molc.91.1:01}

\fullcite{molc.91.1:02}

\fullcite{molc.91.2:01}

\fullcite{molc.91.3:01}

\fullcite{molc.91.3:02}

\fullcite{molc.91.4:01}

\fullcite{molc.91.5:01}

\fullcite{molc.91.6:01}

\end{quotation}

\subsubsection{德文法规}


\begin{quotation}
\fullcite{molc.92:01}

\fullcite{molc.92:02}

\fullcite{molc.92:03}

\end{quotation}


\subsubsection{德文案例}

\begin{quotation}
\fullcite{molc.93:01}

\fullcite{molc.93:02}

\fullcite{molc.93:03}

\end{quotation}

\subsubsection{德文网络文章}

\begin{quotation}
\fullcite{molc.94:01}

\end{quotation}

\subsubsection{几种情况的处理}


\subsection{日文引注体例}

\subsubsection{日文学术文献}

\begin{quotation}
\fullcite{molc.96.1:01}

\fullcite{molc.96.1:02}

\fullcite{molc.96.2:01}

\fullcite{molc.96.3:01}

\end{quotation}

\subsubsection{日文案例}
\begin{quotation}
\fullcite{molc.97:01}

\fullcite{molc.97:02}
\end{quotation}

\subsubsection{日文法规}
\begin{quotation}
\fullcite{molc.98:01}
\end{quotation}
\subsubsection{日本官方文件}
\begin{quotation}
\fullcite{molc.99:01}

\fullcite{molc.99:99}
\end{quotation}
\subsubsection{日文网络文献}

\end{document}
